\documentclass[12pt]{article}

\usepackage{listings}
\usepackage{mathtools}
\usepackage{amsmath}
\usepackage{amssymb}
\usepackage{capt-of}
\usepackage{minted}

% effects
\newcommand\Eff[0]{E}
\newcommand\eff[0]{\varepsilon}
\newcommand\Op[0]{O}
\newcommand\op[0]{op}
\newcommand\inst[0]{\iota}

\newcommand\pty[1]{\ty^1_{#1}}
\newcommand\rty[1]{\ty^2_{#1}}

% types
\newcommand\ty[0]{\tau}
\newcommand\tunit[0]{()}
\newcommand\tarr[2]{#1 \rightarrow #2}
\newcommand\thandler[2]{#1 \Rightarrow #2}

% computation type
\newcommand\cty[0]{\underline{\ty}}
\newcommand\aty[2]{#1 \; ! \; #2}
\newcommand\texists[3]{\exists(#1:#2) . #3}

% values
\newcommand\val[0]{\nu}
\newcommand\vunit[0]{()}
\newcommand\vabs[2]{\lambda #1 . #2}
\newcommand\vhandler[1]{\textit{handler} \; \{#1\}}
\newcommand\vhandleri[2]{\textit{handler} ( #1 ) \; \{#2\}}
\newcommand\vhandlerc[0]{\vhandler{
	\textit{return} \; x \rightarrow \comp,
	\op_1(x ; k) \rightarrow \comp,
	...,
	\op_n(x ; k) \rightarrow \comp
}}
\newcommand\vhandlerci[1]{\vhandleri{#1}{
	\textit{return} \; x \rightarrow \comp,
	\op_1(x ; k) \rightarrow \comp,
	...,
	\op_n(x ; k) \rightarrow \comp
}}

% computations
\newcommand\comp[0]{c}
\newcommand\creturn[1]{\textit{return} \; #1}
\newcommand\capp[2]{#1 \; #2}
\newcommand\cdo[3]{#1 \leftarrow #2 ; #3}
\newcommand\cop[4]{#1(#2 ; \lambda #3 . #4)}
\newcommand\copi[5]{#1 \# #2(#3 ; \lambda #4 . #5)}
\newcommand\chandle[2]{\textit{with} \; #1 \; \textit{handle} \; #2}
\newcommand\cnew[1]{\textit{new} \; #1}

% misc
\newcommand\subty[2]{#1 <: #2}

\lstset{
	frame = single,
	basicstyle=\ttfamily\footnotesize,
	breaklines=true
}

\title{An effect system for dynamic instances}
\author{Albert ten Napel}
\date{}

\begin{document}
\maketitle

\section{Basic effect handler effect system}

The following system is a simplification of the effect system described by Bauer and Pretnar in \cite{effectsystem}.
Instances are removed and handlers can only handle a single effect.
Handlers also should always handle every operations of an effect.
In the effect annotations we will keep track of effect and not operations.

\subsection{Syntax}
We assume there is set of effect names $\Eff = \{ \eff_1, ..., \eff_n \}$.
Each effect has a set of operation names $\Op_\eff = \{ \op_1, ..., \op_n \}$.
We every operation name only corresponds to a single effect.
Each operation has a parameter type $\pty{\op}$ and a return type $\rty{\op}$.
Annotations $e$ are a subset of $\Eff$.
\\
\begin{align*}
	\ty \Coloneqq 	& 											\tag{value types} \\
									& \tunit								\tag{unit type} \\
									& \tarr{\ty}{\cty}			\tag{type of functions} \\
									& \thandler{\cty}{\cty}	\tag{type of handlers} \\
	\cty \Coloneqq 	& 											\tag{computation types} \\
									& \aty{\ty}{e}					\tag{annotated type} \\
	\val \Coloneqq	&												\tag{values} \\
									& x, y, z, k						\tag{variables} \\
									& \vunit								\tag{unit value} \\
									& \vabs{x}{\comp}				\tag{abstraction} \\
									& \vhandlerc						\tag{handler} \\
	\comp \Coloneqq	&												\tag{computations} \\
									& \creturn{\val}				\tag{return value as computation} \\
									& \capp{\val}{\val}			\tag{application} \\
									& \cdo{x}{\comp}{\comp}	\tag{sequencing} \\
									& \cop{\op}{\val}{x}{\comp}	\tag{operation call} \\
									& \chandle{\val}{\comp}	\tag{handler application} \\
\end{align*}

\subsection{Subtyping rules}

\begin{minipage}{0.5\textwidth}
\[\frac{
}{
	\subty{\tunit}{\tunit}
}\]
\vspace{10pt}
\end{minipage}
\begin{minipage}{0.5\textwidth}
\[\frac{
	\begin{array}{l}
	\subty{a'}{a} \\
	\subty{b}{b'}
	\end{array}
}{
	\subty{\tarr{a}{b}}{\tarr{a'}{b'}}
}\]
\vspace{10pt}
\end{minipage}
\begin{minipage}{0.5\textwidth}
\[\frac{
	\begin{array}{l}
	\subty{a'}{a} \\
	\subty{b}{b'}
	\end{array}
}{
	\subty{\thandler{a}{b}}{\thandler{a'}{b'}}
}\]\vspace{10pt}
\end{minipage}
\begin{minipage}{0.5\textwidth}
\[\frac{
	\begin{array}{l}
	\subty{a}{a'} \\
	e \subseteq e'
	\end{array}
}{
	\subty{\aty{a}{e}}{\aty{a'}{e'}}
}\]\vspace{10pt}
\end{minipage}

The following rule says that we can add effects to both sides of a handler type as long as these effects are the same.
This is necessary when applying a handler to a computation that has more effects than the handler handles.
With this subtyping rule we can simply extend the type knowing that the effects will appear on the right-hand side of the handler type and so the extra effects will not become lost.
\[\frac{
}{
	\subty{\thandler{\aty{a}{e_1}}{\aty{b}{e_2}}}{\thandler{\aty{a}{(e_1 \cup e)}}{\aty{b}{(e_2 \cup e)}}}
}\]

\subsection{Typing rules}
For the typing rules there are two judgments,
$\Gamma \vdash \val : \ty$ for assigning types to values and $\Gamma \vdash \comp : \cty$ for assigning computation types to computations. $\Gamma$ stores bindings of variables to types.

\begin{minipage}{0.25\textwidth}
\[\frac{
	\begin{array}{l}
	\Gamma \vdash \val : \ty_1 \\
	\subty{\ty_1}{\ty_2}
	\end{array}
}{
	\Gamma \vdash \val : \ty_2
}\]
\end{minipage}
\begin{minipage}{0.25\textwidth}
\[\frac{
}{
	\Gamma, x : \ty \vdash x : \ty
}\]
\end{minipage}
\begin{minipage}{0.25\textwidth}
\[\frac{
}{
	\Gamma \vdash \vunit : \tunit
}\]
\end{minipage}
\begin{minipage}{0.25\textwidth}
\[\frac{
	\begin{array}{l}
	\Gamma, x : \ty_1 \vdash \comp : \cty_2
	\end{array}
}{
	\Gamma \vdash \vabs{x}{\comp} : \tarr{\ty_1}{\cty_2}
}\]
\end{minipage}

\vspace{10pt}
In the following rule \\$h = 
	\vhandler{
		\textit{return} \; x_r \rightarrow \comp_r,
		\op_1(x_1 ; k_1) \rightarrow \comp_1,
		...,
		\op_n(x_n ; k_n) \rightarrow \comp_n
	}$.
\[\frac{
	\begin{array}{l}
	\Op_\eff = \{ op_1, ..., op_n \} \\
	\Gamma, x_r : \ty_1 \vdash \comp_r : \cty_2 \\
	\Gamma, x_i : \pty{\op_i}, k_i : \tarr{\rty{\op_i}}{\cty_2} \vdash \comp_i : \cty_2 \\
	\end{array}
}{
	\Gamma \vdash h :
	\thandler{\aty{\ty_1}{\{\eff\}}}{\cty_2}
}\]

%%%

\begin{minipage}{0.5\textwidth}
\[\frac{
	\begin{array}{l}
	\Gamma \vdash \comp : \cty_1 \\
	\subty{\cty_1}{\cty_2}
	\end{array}
}{
	\Gamma \vdash \comp : \cty_2
}\]
\end{minipage}
\begin{minipage}{0.5\textwidth}
\[\frac{
	\Gamma \vdash \val : \ty
}{
	\Gamma \vdash \creturn{\val} : \aty{\ty}{\varnothing}
}\]
\end{minipage}
\\
\begin{minipage}{0.333\textwidth}
\[\frac{
	\begin{array}{l}
	\Gamma \vdash \val_1 : \tarr{\ty_1}{\cty_2} \\
	\Gamma \vdash \val_2 : \ty_1
	\end{array}
}{
	\Gamma \vdash \capp{\val_1}{\val_2} : \cty_2
}\]
\end{minipage}
\begin{minipage}{0.333\textwidth}
\[\frac{
	\begin{array}{l}
	\Gamma \vdash \comp_1 : \aty{\ty_1}{e} \\
	\Gamma, x : \ty_1 \vdash \comp_2 : \aty{\ty_2}{e}
	\end{array}
}{
	\Gamma \vdash \cdo{x}{\comp_1}{\comp_2} : \aty{\ty_2}{e}
}\]
\end{minipage}
\begin{minipage}{0.333\textwidth}
\[\frac{
	\begin{array}{l}
	\Gamma \vdash \val : \thandler{\cty_1}{\cty_2} \\
	\Gamma \vdash \comp : \cty_1 \\
	\end{array}
}{
	\Gamma \vdash \chandle{\val}{\comp} : \cty_2
}\]
\end{minipage}
\\
\vspace{10pt}
\[\frac{
	\begin{array}{l}
	\op \in \Op_\eff\\
	\Gamma \vdash \val : \pty{\op} \\
	\Gamma , x : \rty{\op} \vdash \comp : \aty{\ty}{e} \\
	\eff \in e
	\end{array}
}{
	\Gamma \vdash \cop{\op}{\val}{x}{\comp} : \aty{\ty}{e}
}\]

\newpage
\section{Dynamic instances}

\subsection{Syntax}
We extend the system from the previous section with dynamic instances.
We add instances variables to the types.
Annotation $r$ in this system are sets of instance variables.
We add existential quantifiers to the computation types.
Handlers now take an instance as argument.
Operations are called on instances.
We add a new computation to create fresh instances.

\begin{align*}
	\ty \Coloneqq 	& ...											\tag{extended value types} \\
									& i, j, k									\tag{instance variables} \\
	\cty \Coloneqq 	& 												\tag{computation types} \\
									& \aty{\ty}{r}						\tag{annotated type} \\
									& \texists{i}{\eff}{\cty}	\tag{existential} \\
	\val \Coloneqq	&													\tag{updated values} \\
									& \vhandlerci{x}					\tag{handler} \\
	\comp \Coloneqq	&	...											\tag{updated/extended computations} \\
									& \copi{x}{\op}{\val}{y}{\comp}	\tag{operation call} \\
									& \cnew{\eff}							\tag{instance creation} \\
\end{align*}

\subsection{Subtyping}
\begin{minipage}{0.5\textwidth}
\[\frac{
}{
	\subty{i}{i}
}\]
\end{minipage}
\begin{minipage}{0.5\textwidth}
\[\frac{
	\subty{a}{b}
}{
	\subty{\texists{i}{\eff}{a}}{\texists{i}{\eff}{b}}
}\]
\end{minipage}

\newpage
\section{Typing rules}
We update the judgments to include an environment for the instance variables:
$\Delta;\Gamma \vdash \val : \ty$ and $\Delta;\Gamma \vdash \val : \ty$.
$\Delta$ contains bindings of instance variables to effects.

In the following rule \\$h = \vhandleri{x}{
		\textit{return} \; x_r \rightarrow \comp_r,
		\op_1(x_1 ; k_1) \rightarrow \comp_1,
		...,
		\op_n(x_n ; k_n) \rightarrow \comp_n
	}$.
\[\frac{
	\begin{array}{l}
	\Op_\eff = \{ op_1, ..., op_n \} \\
	\Delta;\Gamma \vdash x : i \\
	\Delta \vdash i : \eff \\
	\Delta;\Gamma, x_r : \ty_1 \vdash \comp_r : \cty_2 \\
	\Delta;\Gamma, x_i : \pty{\op_i}, k_i : \tarr{\rty{\op_i}}{\cty_2} \vdash \comp_i : \cty_2 \\
	\end{array}
}{
	\Delta;\Gamma \vdash h :
	\thandler{\aty{\ty_1}{\{i\}}}{\cty_2}
}\]

\begin{minipage}{0.5\textwidth}
\[\frac{
	\begin{array}{l}
	\op \in \Op_\eff\\
	\Delta;\Gamma \vdash x : i \\
	\Delta \vdash i : \eff \\
	\Delta;\Gamma \vdash \val : \pty{\op} \\
	\Delta;\Gamma , y : \rty{\op} \vdash \comp : \aty{\ty}{r} \\
	i \in r
	\end{array}
}{
	\Delta;\Gamma \vdash \copi{x}{\op}{\val}{y}{\comp} : \aty{\ty}{r}
}\]
\vspace{10pt}
\end{minipage}
\begin{minipage}{0.5\textwidth}
\[\frac{
	\begin{array}{l}
	i\textnormal{ is a fresh instance variable}
	\end{array}
}{
	\Delta;\Gamma \vdash \cnew{\eff} : \aty{i}{\varnothing}
}\]
\vspace{10pt}
\end{minipage}
\\
\begin{minipage}{0.5\textwidth}
\[\frac{
	\Delta,i:\eff;\Gamma \vdash \comp : \cty
}{
	\Delta;\Gamma \vdash \comp : \texists{i}{\eff}{\cty}
}\]
\end{minipage}
\begin{minipage}{0.5\textwidth}
\[\frac{
	\Delta;\Gamma \vdash \comp : \texists{i}{\eff}{\cty}
}{
	\Delta;\Gamma \vdash \comp : \cty
}\]
\end{minipage}

\section{Example}

Given a boolean type $Bool$ and an effect $Flip$ with one operation\\ $flip : \tarr{\tunit}{Bool}$.
The following short example results in the derivations below.

\begin{minted}[tabsize=2]{haskell}
\() ->
	x <- new Flip;
	x#flip ()
\end{minted}

\[\frac{
	\cdot;u:\tunit \vdash \cdo{x}{\cnew{Flip}}{\copi{x}{flip}{\vunit}{y}{\creturn{y}}} :
	\texists{i}{Flip}{\aty{Bool}{\{i\}}}
}{
	\cdot;\cdot \vdash \vabs{u}{\cdo{x}{\cnew{Flip}}{\copi{x}{flip}{\vunit}{y}{\creturn{y}}}}
	: \tarr{\tunit}{\texists{i}{Flip}{\aty{Bool}{\{i\}}}}
}\]

\[
\frac{
	i:Flip;u:\tunit \vdash \cdo{x}{\cnew{Flip}}{\copi{x}{flip}{\vunit}{y}{\creturn{y}}} :
	\aty{Bool}{\{i\}}
	}{
	\cdot;u:\tunit \vdash \cdo{x}{\cnew{Flip}}{\copi{x}{flip}{\vunit}{y}{\creturn{y}}} :
	\texists{i}{Flip}{\aty{Bool}{\{i\}}}
	}
\]

\[\frac{
\begin{array}{l}
	i:Flip;u:\tunit \vdash \cnew{Flip} : \aty{i}{\{i\}} \\
	i:Flip;u:\tunit,x:i \vdash \copi{x}{flip}{\vunit}{y}{\creturn{y}} : \aty{Bool}{\{i\}} \\
\end{array}
}{
i:Flip;u:\tunit \vdash \cdo{x}{\cnew{Flip}}{\copi{x}{flip}{\vunit}{y}{\creturn{y}}} :
	\aty{Bool}{\{i\}}
}\]

\[\frac{
\begin{array}{l}
i:Flip;u:\tunit \vdash \cnew{Flip} : \aty{i}{\varnothing} \\
\subty{\aty{i}{\varnothing}}{\aty{i}{\{i\}}}
\end{array}
}{
i:Flip;u:\tunit \vdash \cnew{Flip} : \aty{i}{\{i\}}
}\]

\[\frac{
\begin{array}{l}
i:Flip;u:\tunit,x:i \vdash x : i \\
i:Flip \vdash i : Flip \\
i:Flip;u:\tunit,x:i \vdash \vunit : \tunit \\
i:Flip;u:\tunit,x:i,y:Bool \vdash \creturn{y} : \aty{Bool}{\{i\}} \\
\end{array}
}{
i:Flip;u:\tunit,x:i \vdash \copi{x}{flip}{\vunit}{y}{\creturn{y}} : \aty{Bool}{\{i\}}
}\]

\[\frac{
\begin{array}{l}
i:Flip;u:\tunit,x:i,y:Bool \vdash \creturn{y} : \aty{Bool}{\varnothing} \\
\subty{\aty{Bool}{\varnothing}}{\aty{Bool}{\{i\}}}
\end{array}
}{
i:Flip;u:\tunit,x:i,y:Bool \vdash \creturn{y} : \aty{Bool}{\{i\}}
}\]

Similarly the following program:
\begin{minted}[tabsize=2]{haskell}
\() ->
	inst <- new Exception;
	(
		\x -> if x == 0 then inst#throw () else x,
		\f -> handler(inst) { throw () k -> f () }
	)
:
() -> exists (i : Exception).
	(
		Int -> Int!{i},
		(() -> t!e) -> (Int!{i} => t!e)
	)
\end{minted}

\section{Alternative rules for existentials}

Another idea is to have $\cnew{\eff}$ return an existential type and to unpack existentials only when sequencing.
If the instance is mentioned in the return type of the sequencing we have to repack the existential.

\[\frac{
}{
	\Delta;\Gamma \vdash \cnew{\eff} : \texists{i}{\eff}{\aty{i}{\varnothing}}
}\]

\begin{minipage}{0.5\textwidth}
\[\frac{
	\begin{array}{l}
	\Delta;\Gamma \vdash \comp_1 : \texists{i}{\eff}{\aty{\ty_1}{e}} \\
	\Delta,i:\eff;\Gamma,x:\ty_1 \vdash \comp_2 : \aty{\ty_2}{e} \\
	i \notin \ty_2
	\end{array}
}{
	\Delta;\Gamma \vdash \cdo{x}{\comp_1}{\comp_2} : \aty{\ty_2}{e}
}\]
\end{minipage}
\begin{minipage}{0.5\textwidth}
\[\frac{
	\begin{array}{l}
	\Delta;\Gamma \vdash \comp_1 : \texists{i}{\eff}{\aty{\ty_1}{e}} \\
	\Delta,i:\eff;\Gamma,x:\ty_1 \vdash \comp_2 : \aty{\ty_2}{e} \\
	i \in \ty_2
	\end{array}
}{
	\Delta;\Gamma \vdash \cdo{x}{\comp_1}{\comp_2} : \texists{i}{\eff}{\aty{\ty_2}{e}}
}\]
\end{minipage}

\begin{thebibliography}{9}
\bibitem{effectsystem}
Bauer, Andrej, and Matija Pretnar. ``An effect system for algebraic effects and handlers.'' International Conference on Algebra and Coalgebra in Computer Science. Springer, Berlin, Heidelberg, 2013.
\end{thebibliography}

\end{document}
