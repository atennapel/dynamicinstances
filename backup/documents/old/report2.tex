\documentclass[12pt]{article}

\usepackage{listings}
\usepackage{mathtools}

% expressions
\newcommand\eapp[2]{#1 \; #2}
\newcommand\eimplapp[2]{#1 \; \{ #2 \} }
\newcommand\eabs[2]{\lambda #1 \; . \; #2}
\newcommand\eabsanno[3]{\lambda(#1 : #2) \; . \; #3}
\newcommand\eimplabs[2]{\lambda \{ #1 \} \; . \; #2}
\newcommand\eanno[2]{#1 : #2}
\newcommand\elet[3]{$let$ \; #1 = #2 \; $in$ \; #3}
\newcommand\eletr[3]{$letr$ \; #1 = #2 \; $in$ \; #3}

\lstset{
	frame = single,
	basicstyle=\ttfamily\footnotesize,
	breaklines=true
}

\title{Thesis report 1b}
\author{Albert ten Napel}
\date{}

\begin{document}

\section{Syntax}

\begin{tabular}{l l l}
	$\kappa \Coloneqq$ & & kinds \\
		& Type & kind of values \\
		& Row & kind of rows \\
		& Label & kind of labels \\
		& $\kappa \rightarrow \kappa$ & kind arrow \\
\end{tabular}

\vspace{15pt}\noindent
$l$ ranges of an infinite set of labels \\

\vspace{15pt}\noindent
\begin{tabular}{l l l}
	$\pi \Coloneqq$ & & constraints \\
		& $\tau / l$ & row lacks label \\
		& value($\tau$) & type is not an effect type \\
		& $\tau \sim \tau$ & row equality \\
\end{tabular}

\vspace{15pt}\noindent
\begin{tabular}{l l l}
	$\sigma \Coloneqq$ & & constrained type \\
		& $\pi \Rightarrow \sigma$ & constraint \\
		& $\tau$ & type \\
\end{tabular}

\vspace{15pt}\noindent
\begin{tabular}{l l l}
	$\tau \Coloneqq$ & & types \\
		& $l$ & label \\
		& $c$ & type constructor \\
		& $\alpha$ & type variable\\
		& $\tau \; \tau$ & type application \\
		& $<>$ & empty row \\ 
		& $< \; $l$ \; : \tau \; | \; \tau >$ & row extension \\
		& $\mu \alpha \; . \; \tau$ & recursive type \\
		& $\forall \alpha \; . \; \sigma$ & forall with constraint \\
\end{tabular}

\vspace{15pt}\noindent
\begin{tabular}{l l l}
	$e \Coloneqq$ & & terms \\
		& $'l$ & label \\
		& $x$ & variable \\
		& $\eapp{e}{e}$ & application \\
		& $\eabs{x}{e}$ & abstraction \\
		& $\eabsanno{x}{\tau}{e}$ & annotated abstraction \\
		& $\elet{x}{e}{e}$ & let expression \\
		& $\eletr{x}{e}{e}$ & recursive let expression \\

		& $\{\}$ & empty record \\

		& handle $\{$ return $x \rightarrow x, l \; x \; k \rightarrow e, ... \}$ & effect handling \\
\end{tabular}

\subsection{Type aliases}
() $:$ Rec $<>$ \\

\subsection{Type constructors}
Lab $:$ Label $\rightarrow$ Type \\
Rec $:$ Row $\rightarrow$ Type \\
Var $:$ Row $\rightarrow$ Type \\
Eff $:$ Row $\rightarrow$ Type $\rightarrow$ Type \\

\subsection{Constants}
() $:$ Rec $<>$ \hspace{15pt} unit \\
$(.) : \forall l t r \; . \; r/l  \; \Rightarrow \; $Lab$ \; l \rightarrow $Rec$ < l : t \; | \; r > \rightarrow t $ \hspace{15pt} record selection \\
$(.{+}) : \forall l t r \; . \; r/l  \; \Rightarrow \; $Lab$ \; l \rightarrow $Rec$ \; r \rightarrow $Rec$ < l : t \; | \; r > $ \hspace{15pt} record extension \\
$(.-) : \forall l t r \; . \; r/l  \; \Rightarrow \; $Lab$ \; l \rightarrow $Rec$ <l : t \; | \; r > \rightarrow $Rec$ \; r $ \hspace{15pt} record restriction \\
$(@) : \forall l t r \; . \; r/l  \; \Rightarrow \; $Lab$ \; l \rightarrow t \rightarrow $Var$ <l : t \; | \; r >$ \hspace{15pt} variant injection \\
$(@{+}) : \forall l t r \; . \; r/l  \; \Rightarrow \; $Lab$ \; l \rightarrow $Var$ \; r \rightarrow $Var$ <l : t \; | \; r >$ \hspace{15pt} variant embedding \\
$(?) : \forall l a b r \; . \; r/l  \; \Rightarrow \; $Lab$ \; l \rightarrow ( a \rightarrow b ) \rightarrow ($Var$ \; r \rightarrow b) \rightarrow $Var$ <l : a \; | \; r> \rightarrow b$ \hspace{15pt} variant elimination \\
end $: \forall t \; . \; $Var$ <> \rightarrow t$ \hspace{15pt} end variant elimination \\
$(!) : \forall l a b r \; . \; r/l  \; \Rightarrow \; $Lab$ \; l \rightarrow a \rightarrow $Eff$ <l : a \rightarrow b \; | \; r > b$ \hspace{15pt} perform effect \\
bind $: \forall a b r \; . \; $Eff$ \; r \; a \rightarrow (a \rightarrow $Eff$ \; r \; b) \rightarrow $Eff$ \; r \; b $ \hspace{15pt} effect sequencing \\
$(!{+}) : \forall l a b t r \; . \; r/l  \; \Rightarrow \; $Lab$ \; l \rightarrow $Eff$ \; r \; t \rightarrow $Eff$ <l : a \rightarrow b \; | \; r > t$ \hspace{15pt} effect embedding \\
pure $: \forall rt \; . \; $Eff$ \; r \; t  \rightarrow t$ \hspace{15pt} value from effect type \\
return $: \forall rt \; . \; t \rightarrow $Eff$ \; r \; t$ \hspace{15pt} value into effect type \\

\section{Typing rules}
Var \[\frac{
	x : \tau \in \Gamma
}{
	\Gamma \vdash x : \tau
}\] \\
Gen \[\frac{
	\begin{array}{c}
	\Gamma \vdash e : \tau \\
	\alpha \notin ftv(\Gamma)
	\end{array}
}{
	\Gamma \vdash e : \forall \alpha . \tau
}\]\\
Inst \[\frac{
	\begin{array}{c}
	\Gamma \vdash e : \tau_1 \\
	\tau_1 \sqsubseteq \tau_2
	\end{array}
}{
	\Gamma \vdash e : \tau_2
}\]\\
Fun \[\frac{
	\Gamma , x : \tau \vdash e : \rho
}{
	\Gamma \vdash \eabs{x}{e} : \tau \rightarrow \rho
}\]\\
Fun-Ann \[\frac{
	\Gamma , x : \tau \vdash e : \rho
}{
	\Gamma \vdash \eabsanno{x}{\tau}{e} : \tau \rightarrow \rho
}\]\\
Let \[\frac{
	\begin{array}{c}
	\Gamma \vdash e_1 : \tau_1 \\
	\Gamma , x : \tau_1 \vdash e_2 : \tau_2 \\
	\forall \tau_1^\prime . \Gamma \vdash e_1 : \tau_1^\prime \Rightarrow \tau_1 \sqsubseteq \tau_1^\prime
	\end{array}
}{
	\Gamma \vdash let \; x = e_1 \; in \; e_2 : \tau_2
}\]\\
Letr \[\frac{
	\begin{array}{c}
	\Gamma , x : \alpha \vdash e_1 : \tau_1 \\
	\alpha \sqsubseteq \tau_1 \\
	\Gamma , x : \tau_1 \vdash e_2 : \tau_2 \\
	\forall \tau_1^\prime . \Gamma \vdash e_1 : \tau_1^\prime \Rightarrow \tau_1 \sqsubseteq \tau_1^\prime
	\end{array}
}{
	\Gamma \vdash letr \; x = e_1 \; in \; e_2 : \tau_2
}\]\\
App \[\frac{
	\begin{array}{c}
	\Gamma \vdash e_1 : \tau_2 \rightarrow \tau \\
	\Gamma \vdash e_2 : \tau_2 \\
	(\forall \tau^\prime \tau_2^\prime . (\Gamma \vdash e_1 : \tau_2^\prime \rightarrow \tau^\prime \wedge \Gamma \vdash e_2 : \tau_2^\prime)\\
	\Rightarrow [\![ \tau_2 \rightarrow \tau ]\!] \leq [\![ \tau_2^\prime \rightarrow \tau^\prime ]\!])
	\end{array}
}{
	\Gamma \vdash \eapp{e_1}{e_2} : \tau
}\]\\

\newpage
\section{Examples}

\begin{lstlisting}[caption=Records and variants]
.'x { x = 10 } == 10
.+'x 10 {} == { x = 10 }
.-'x { x = 10 } == {}
.:'x (\x -> x + 1) { x = 10 } == { x = 11 }

@'Just 10 == @'Just 10
@:'Just (\x -> x + 1) (@'Just 10) == @'Just 11
?'Just (\x -> x + 1) (\_ -> 0) (@'Just 10) == 11
?'Just (\x -> x + 1) (\_ -> 0) (@'Nothing {}) == 0

?'Just (\x -> x + 1) (\_ -> 0)
  : forall (r : Row) . Var < Just : Int | r > -> Int
?'Just (\x -> x + 1) end : Var < Just : Int > -> Int
\end{lstlisting}

\begin{lstlisting}[caption=Basic effect handlers]
// define flip action
flip : Eff < Flip : () -> Bool | r > Bool
flip = !Flip {}

// program that uses the flip effect
program : Eff < Flip : () -> Bool > Bool
program =
  x <- flip;
  y <- flip;
  return (x || y)

// handler that always returns True
alwaysTrue : Eff < Flip : () -> Bool | r > t -> Eff r t
alwaysTrue = handle { Flip {} k _ -> k True _ } ()

// result of the program
result : Bool
result = pure (alwaysTrue program) == True
\end{lstlisting}

\newpage
\begin{lstlisting}[caption=State effect]
// state effect handler (v = initial state)
state :
  v ->
  Eff < Get : () -> v, Set : v -> () | r > t ->
  Eff r t
state = handle {
  Get _ k v -> k v v)
  Set v k _ -> k () v)
}

get = !Get {}
set v = !Set v

program =
  x <- get;
  set 10;
  y <- get;
  return (x + y)

result = pure (state 100 program) == 110
\end{lstlisting}

\begin{lstlisting}[caption=IO effects]
runIO :
  Eff <
    putLine : Str -> (),
    getLine : () -> Str
    | r
  > t -> Eff r t

infiniteGreeter =
  name <- getLine;
  putLine ("Hello " ++ name ++ "!");
  infiniteGreeter

main = runIO infiniteGreeter
\end{lstlisting}

\newpage
\begin{lstlisting}[caption=Recursion effect]
// fix as an effectful function
fix : (t -> t) -> Eff < fix : (t -> t) -> t | r > t
runFix : Eff < fix : (t -> t) -> t | r > x -> Eff r x 

facEff : Eff < fix : (Int -> Int) -> Int | r > (Int -> Int)
facEff = fix (\fac n ->
  if (n <= 1)
    1
  else
    n * (fac (n - 1)))

fac : Int -> Eff < fix : (Int -> Int) -> Int | r > Int
fac n =
  f <- fac;
  return (f n)

main = runFix (fac 10)
\end{lstlisting}

\newpage
\section{Questions}
- How does impredicativity interact with row-polymorphic types or algebraic effects? \\
- How do higher-ranked types interact with row-polymorphic types or algebraic effects. (ST monad in Haskell?) \\
- Handlers that only handle one effect? (hypothesis: not as expressive as handlers with multiple effects) \\
- Best way to introduce recursive types in to the system? (equi-recursive or iso-recursive) \\
- Is a seperation between value types and computation types necessary? (call-by-push-value) \\

\section{Papers}

\subsection{Type system}
\textbf{Generalizing Hindley-Milner type inference algorithms} \\
\textit{Heeren, B. J., Jurriaan Hage, and S. Doaitse Swierstra. "Generalizing Hindley-Milner type inference algorithms." (2002).} \\
Description of the Hindley-Milner type system and inference algorithm. Also describes a constraint-solving algorithm.
\\
\textbf{HMF: Simple type inference for first-class polymorphism} \\
\textit{Leijen, Daan. "HMF: Simple type inference for first-class polymorphism." ACM Sigplan Notices. Vol. 43. No. 9. ACM, 2008.} \\
Describes an extension of Hindley-Milner that enables System F types including rank-N types and impredicative polymorphism.
\\
\textbf{Complete and easy bidirectional typechecking for higher-rank polymorphism.} \\
\textit{Dunfield, Joshua, and Neelakantan R. Krishnaswami. "Complete and easy bidirectional typechecking for higher-rank polymorphism." ACM SIGPLAN Notices. Vol. 48. No. 9. ACM, 2013.} 
A type system with System F types, including higher-ranked types and predicative instantiation. Contains bidirectional typing rules. Can subsume Hindley-Milner.
\\
\textbf{Ur: statically-typed metaprogramming with type-level record computation.} \\
\textit{Chlipala, Adam. "Ur: statically-typed metaprogramming with type-level record computation." ACM Sigplan Notices. Vol. 45. No. 6. ACM, 2010.} \\
Describes the programming Language Ur, which has advanced type-level computation on records.

\subsection{Row polymorphism}
\textbf{A polymorphic type system for extensible records and variants} \\
\textit{Gaster, Benedict R., and Mark P. Jones. "A polymorphic type system for extensible records and variants." (1996).} \\
Describes a simple type system with (row polymorphic) extensible records and variants that only require lacks constraints.
\\
\textbf{Extensible records with scoped labels.} \\
\textit{Leijen, Daan. "Extensible records with scoped labels." Trends in Functional Programming 5 (2005): 297-312.} \\
Describes a very simple extension to Hindley-Milner that support extensible records and "scoped labels", which means labels can occur multiple times in a row.
This requires no constraints at all.
\\
\textbf{First-class labels for extensible rows.} \\
\textit{Leijen, D. J. P. "First-class labels for extensible rows." (2004).} \\
Describes a type system where labels are first-class and one can define functions that take labels as arguments.
This simplifies the language but complicates the type system.

\subsection{Effect handlers}
\textbf{An effect system for algebraic effects and handlers.} \\
\textit{Bauer, Andrej, and Matija Pretnar. "An effect system for algebraic effects and handlers." International Conference on Algebra and Coalgebra in Computer Science. Springer, Berlin, Heidelberg, 2013.} \\
Describes an effect system called "core Eff" which is an extension of a ML-style language with algebraic effects and handlers. The system is formalized in Twelf.
\\
\textbf{Programming with algebraic effects and handlers.} \\
\textit{Bauer, Andrej, and Matija Pretnar. "Programming with algebraic effects and handlers." Journal of Logical and Algebraic Methods in Programming 84.1 (2015): 108-123.} \\
Describes the programming language Eff, which is a ML-like language with algebraic effects and effect handlers.
\\
\textbf{An introduction to algebraic effects and handlers.} \\
\textit{Pretnar, Matija. "An introduction to algebraic effects and handlers. invited tutorial paper." Electronic Notes in Theoretical Computer Science 319 (2015): 19-35.} \\
This paper is a nice introduction to algebraic effects and handlers. It shows examples and gives semantics and typing rules.
\\
\textbf{Liberating effects with rows and handlers.} \\
\textit{Hillerstr\"{o}m, Daniel, and Sam Lindley. "Liberating effects with rows and handlers." Proceedings of the 1st International Workshop on Type-Driven Development. ACM, 2016.} \\
Describes the Links programming language, which combines algebraic effects and handlers with row polymorphism. Includes a formalization.
\\
\textbf{Algebraic effects and effect handlers for idioms and arrows.} \\
\textit{Lindley, Sam. "Algebraic effects and effect handlers for idioms and arrows." Proceedings of the 10th ACM SIGPLAN workshop on Generic programming. ACM, 2014.} \\
Describes a generalization of algebraic effects that allows for other kinds of effectful computations.
\\
\textbf{Koka: Programming with row polymorphic effect types.} \\
\textit{Leijen, Daan. "Koka: Programming with row polymorphic effect types." arXiv preprint arXiv:1406.2061 (2014).} \\
Describes a programming language called Koka that has row polymorphic effect types.
\\
\textbf{Type directed compilation of row-typed algebraic effects.} \\
\textit{Leijen, Daan. "Type directed compilation of row-typed algebraic effects." POPL. 2017.} \\
Provides a nice up-to-date presentation of Koka, including algebraic effects and handlers.
\\
\textbf{Do Be Do Be Do: The Frank Programming Language} \\
\textit{Lindley, Sam \& McBride, Conor, "http://homepages.inf.ed.ac.uk/slindley/papers/frankly-draft-march2014.pdf"} \\
Describes a programming language called Frank where every function is an effect handler. Any function will implicitly work over effectful programs.
Makes a distinction between value and computation types.
\end{document}
