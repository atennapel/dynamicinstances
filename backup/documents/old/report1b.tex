\documentclass[12pt]{article}

\usepackage{listings}
\usepackage{mathtools}

% expressions
\newcommand\eapp[2]{#1 \; #2}
\newcommand\eimplapp[2]{#1 \; \{ #2 \} }
\newcommand\eabs[2]{\lambda #1 \; . \; #2}
\newcommand\eimplabs[2]{\lambda \{ #1 \} \; . \; #2}
\newcommand\eanno[2]{#1 : #2}
\newcommand\elet[3]{$let$ \; #1 = #2 \; $in$ \; #3}
\newcommand\eletr[3]{$letr$ \; #1 = #2 \; $in$ \; #3}

\lstset{
	frame = single,
	basicstyle=\ttfamily\footnotesize,
	breaklines=true
}

\title{Thesis report 1b}
\author{Albert ten Napel}
\date{}

\begin{document}

\section{Syntax}

\begin{tabular}{l l l}
	$\kappa \Coloneqq$ & & kinds \\
		& Type & kind of values \\
		& Row & kind of rows \\
		& $\kappa \rightarrow \kappa$ & kind arrow \\
\end{tabular}

\vspace{15pt}\noindent
\begin{tabular}{l l l}
	$\pi \Coloneqq$ & & constraints \\
		& $\tau / l$ & row lacks label \\
		& value($\tau$) & type is not an effect type \\
		& implicit($\tau$) & value of type is in enviroment \\
\end{tabular}

\vspace{15pt}\noindent
\begin{tabular}{l l l}
	$\sigma \Coloneqq$ & & polytypes \\
		& $\forall \bar{\alpha} \; . \; \bar{\pi} \Rightarrow \tau$ & forall with constraints \\
\end{tabular}

\vspace{15pt}\noindent
\begin{tabular}{l l l}
	$\tau \Coloneqq$ & & monotypes \\
		& $c$ & type constructor \\
		& $\alpha$ & type variable\\
		& $\tau \; \tau$ & type application \\
		& $<>$ & empty row \\ 
		& $< \; $l$ \; : \tau \; | \; \tau >$ & row extension \\
		& $\mu \alpha \; . \; \tau$ & recursive type \\
\end{tabular}

\vspace{15pt}\noindent
\begin{tabular}{l l l}
	$e \Coloneqq$ & & terms \\
		& $l$ & label \\
		& $x$ & variable \\
		& $\eapp{e}{e}$ & application \\
		& $\eimplapp{e}{e}$ & implicit application \\
		& $\eabs{x}{e}$ & abstraction \\
		& $\eimplabs{x}{e}$ & implicit abstraction \\
		& $\elet{x}{e}{e}$ & let expression \\
		& $\eletr{x}{e}{e}$ & recursive let expression \\
		& $\eanno{e}{\tau}$ & type annotation \\

		& $\{\}$ & empty record \\
		& $.l$ & record selection \\
		& $.{+}l$ & record extension \\
		& $.{-}l$ & record restriction \\
		& $.{:}l$ & record update \\

		& $@l$ & variant injection \\
		& $@{+}l$ & variant embedding \\
		& $@{:}l$ & variant update \\
		& $?l$ & variant elimination \\
		& $end$ & end variant elimination \\

		& $x \leftarrow e ; e$ & effect sequencing \\
		& $e ; e$ & effect sequencing (discard argument) \\
		& $!l$ & perform effect \\
		& $!{+}l$ & embed effect \\
		& handle $\{$ return $x \rightarrow x, l \; x \; k \rightarrow e, ... \}$ & effect handling \\
		& pure e & unwrap value from effect type \\
		& return e & wrap value in effect type \\
\end{tabular}

\subsection{Type constants}
Rec $:$ Row $\rightarrow$ Type \\
Var $:$ Row $\rightarrow$ Type \\
Eff $:$ Row $\rightarrow$ Type $\rightarrow$ Type \\

\newpage
\section{Examples}

\begin{lstlisting}[caption=Records and variants]
.x { x = 10 } == 10
.+x 10 {} == { x = 10 }
.-x { x = 10 } == {}
.:x (\x -> x + 1) { x = 10 } == { x = 11 }

@Just 10 == @Just 10
@:Just (\x -> x + 1) (@Just 10) == @Just 11
?Just (\x -> x + 1) (\_ -> 0) (@Just 10) == 1
?Just (\x -> x + 1) (\_ -> 0) (@Nothing {}) == 0
\end{lstlisting}

\begin{lstlisting}[caption=Basic effect handlers]
// define flip action
flip : Eff { Flip : {} -> Bool | r } Bool
flip = !Flip {}

// program that uses the flip effect
program : Eff { Flip : {} -> Bool } Bool
program =
  x <- flip;
  y <- flip;
  return (x || y)

// handler that always returns True
alwaysTrue : Eff { Flip : {} -> Bool | r } t -> Eff r t
alwaysTrue = handle { Flip {} k -> k True }

// result of the program
result : Bool
result = pure (alwaysTrue program) == True
\end{lstlisting}

\newpage
\begin{lstlisting}[caption=State effect]
// state effect handler (v = initial state)
state :
  Eff { Get : {} -> v, Set : v -> {} | r } t ->
  Eff r (v -> Eff p t)
state = handle {
  Get _ k -> return (\v -> (f <- k v; f v))
  Set v k -> return (\_ -> (f <- k v; f v))
  return v -> return (\_ -> return v)
}

get = !Get {}
set v = !Set v

program =
  x <- get;
  set 10;
  y <- get;
  return (x + y)

result = pure (pure (state program) 100) == 110
\end{lstlisting}

\begin{lstlisting}[caption=IO effects]
runIO :
  Eff {
    putLine : Str -> {},
    getLine : {} -> Str
    | r
  } t -> Eff r t

infiniteGreeter =
  name <- getLine;
  putLine ("Hello " ++ name ++ "!");
  infiniteGreeter

main = runIO infiniteGreeter
\end{lstlisting}

\newpage
\begin{lstlisting}[caption=Recursion effect]
// fix as an effectful function
fix : (t -> t) -> Eff { fix : (t -> t) -> t | r } t
runFix : Eff { fix : (t -> t) -> t | r} x -> Eff r x 

facEff : Eff { fix : (Int -> Int) -> Int | r } (Int -> Int)
facEff = fix (\fac n ->
  if (n <= 1)
    1
  else
    n * (fac (n - 1)))

fac : Int -> Eff { fix : (Int -> Int) -> Int | r } Int
fac n =
  f <- fac;
  return (f n)

main = runFix (fac 10)
\end{lstlisting}

\begin{lstlisting}[caption=Implicits]
Show : (t -> Str) -> Var { Show : t -> Str }
Show f = @Show f

show : Var { Show : t -> Str } -> (t -> Str)
show {showInstance} = (?Show id end) showInstance

int2string : Int -> Str

showInt : Var { Show : Int -> Str }
showInt = Show int2string

main : forall {} . implicit(Var { Show : Int -> Str }) => Str
main = (show 10) ++ (show {showInt} 10)
\end{lstlisting}

\newpage
\section{Papers}

\subsection{Type system}
\textbf{Generalizing Hindley-Milner type inference algorithms} \\
\textit{Heeren, B. J., Jurriaan Hage, and S. Doaitse Swierstra. "Generalizing Hindley-Milner type inference algorithms." (2002).} \\
Description of the Hindley-Milner type system and inference algorithm. Also describes a constraint-solving algorithm.
\\
\textbf{HMF: Simple type inference for first-class polymorphism} \\
\textit{Leijen, Daan. "HMF: Simple type inference for first-class polymorphism." ACM Sigplan Notices. Vol. 43. No. 9. ACM, 2008.} \\
Describes an extension of Hindley-Milner that enables System F types including rank-N types and impredicative polymorphism.

\subsection{Row polymorphism}
\textbf{A polymorphic type system for extensible records and variants} \\
\textit{Gaster, Benedict R., and Mark P. Jones. "A polymorphic type system for extensible records and variants." (1996).} \\
Describes a simple type system with (row polymorphic) extensible records and variants that only require lacks constraints.
\\
\textbf{Extensible records with scoped labels.} \\
\textit{Leijen, Daan. "Extensible records with scoped labels." Trends in Functional Programming 5 (2005): 297-312.} \\
Describes a very simple extension to Hindley-Milner that support extensible records and "scoped labels", which means labels can occur multiple times in a row.
This requires no constraints at all.
\\
\textbf{First-class labels for extensible rows.} \\
\textit{Leijen, D. J. P. "First-class labels for extensible rows." (2004).} \\
Describes a type system where labels are first-class and one can define functions that take labels as arguments.
This simplifies the language but complicates the type system.

\subsection{Effect handlers}
\textbf{An effect system for algebraic effects and handlers.} \\
\textit{Bauer, Andrej, and Matija Pretnar. "An effect system for algebraic effects and handlers." International Conference on Algebra and Coalgebra in Computer Science. Springer, Berlin, Heidelberg, 2013.} \\
Describes an effect system called "core Eff" which is an extension of a ML-style language with algebraic effects and handlers. The system is formalized in Twelf.
\\
\textbf{Programming with algebraic effects and handlers.} \\
\textit{Bauer, Andrej, and Matija Pretnar. "Programming with algebraic effects and handlers." Journal of Logical and Algebraic Methods in Programming 84.1 (2015): 108-123.} \\
Describes the programming language Eff, which is a ML-like language with algebraic effects and effect handlers.
\\
\textbf{An introduction to algebraic effects and handlers.} \\
\textit{Pretnar, Matija. "An introduction to algebraic effects and handlers. invited tutorial paper." Electronic Notes in Theoretical Computer Science 319 (2015): 19-35.} \\
This paper is a nice introduction to algebraic effects and handlers. It shows examples and gives semantics and typing rules.
\\
\textbf{Liberating effects with rows and handlers.} \\
\textit{Hillerström, Daniel, and Sam Lindley. "Liberating effects with rows and handlers." Proceedings of the 1st International Workshop on Type-Driven Development. ACM, 2016.} \\
Describes the Links programming language, which combines algebraic effects and handlers with row polymorphism. Includes a formalization.
\\
\textbf{Algebraic effects and effect handlers for idioms and arrows.} \\
\textit{Lindley, Sam. "Algebraic effects and effect handlers for idioms and arrows." Proceedings of the 10th ACM SIGPLAN workshop on Generic programming. ACM, 2014.} \\
Describes a generalization of algebraic effects that allows for other kinds of effectful computations.
\\
\textbf{Koka: Programming with row polymorphic effect types.} \\
\textit{Leijen, Daan. "Koka: Programming with row polymorphic effect types." arXiv preprint arXiv:1406.2061 (2014).} \\
Describes a programming language called Koka that has row polymorphic effect types.
\\
\textbf{Do Be Do Be Do: The Frank Programming Language} \\
\textit{Lindley, Sam \& McBride, Conor, "http://homepages.inf.ed.ac.uk/slindley/papers/frankly-draft-march2014.pdf"} \\
Describes a programming language called Frank where every function is an effect handler. Any function will implicitly work over effectful programs.
Makes a distinction between value and computation types.

\subsection{Implicits}
\textbf{The implicit calculus: a new foundation for generic programming} \\
\textit{Oliveira, Bruno CdS, et al. "The implicit calculus: a new foundation for generic programming." ACM SIGPLAN Notices. Vol. 47. No. 6. ACM, 2012.} \\
Describes a formalization of a minimal lambda calculus with implicits.
\\
\textbf{On the bright side of type classes: instance arguments in Agda.} \\
\textit{Devriese, Dominique, and Frank Piessens. "On the bright side of type classes: instance arguments in Agda." ACM SIGPLAN Notices 46.9 (2011): 143-155.} \\
Describes an implementation of implicits in Agda called instance arguments.
\\
\textbf{Modular implicits.} \\
\textit{White, Leo, Frédéric Bour, and Jeremy Yallop. "Modular implicits." arXiv preprint arXiv:1512.01895 (2015).} \\
Describes an extension of OCaml that adds implicits to modules, enabling ad-hoc polymorphism. 
\\
\textbf{Type classes as objects and implicits.} \\
\textit{Oliveira, Bruno CdS, Adriaan Moors, and Martin Odersky. "Type classes as objects and implicits." ACM Sigplan Notices. Vol. 45. No. 10. ACM, 2010.} \\
Describes how to implement type classes in Scala using it's objects and implicits parameters.
\end{document}
