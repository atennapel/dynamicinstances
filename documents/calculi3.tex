\documentclass[12pt]{article}

\usepackage{listings}
\usepackage{mathtools}
\usepackage{amsmath}
\usepackage{amssymb}
\usepackage{capt-of}
\usepackage{minted}

% effects
\newcommand\eff[0]{\varepsilon}
\newcommand\steffs[0]{E}
\newcommand\effs[0]{\eff^*}
\newcommand\eop[0]{\textit{op}}
\newcommand\eopa[0]{\textit{op}^*}
\newcommand\eops[1]{O^{#1}}
\newcommand\allinsts[0]{I}
\newcommand\insts[1]{\allinsts^{#1}}
\newcommand\inst[0]{\iota}
\newcommand\instss[0]{\inst^*}

% types
\newcommand\type[0]{\tau}
\newcommand\tunit[0]{()}
\newcommand\tbool[0]{\textit{Bool}}
\newcommand\tarr[2]{#1 \rightarrow #2}
\newcommand\tarre[3]{#1 \rightarrow #2 \; #3}
\newcommand\thandler[4]{#1 \; #2 \Rightarrow #3 \; #4}
\newcommand\tinst[1]{\textit{Inst} \; #1}

% values
\newcommand\val[0]{\nu}
\newcommand\vunit[0]{()}
\newcommand\vtrue[0]{\textit{true}}
\newcommand\vfalse[0]{\textit{false}}
\newcommand\vabs[2]{\lambda #1 . #2}
\newcommand\vhandler[1]{\textit{handler} \; \{#1\}}
\newcommand\vhandlerc[0]{\vhandler{
	\textit{return} \; x \rightarrow \comp,
	\eop_1(x ; k) \rightarrow \comp,
	...,
	\eop_n(x ; k) \rightarrow \comp
}}
\newcommand\vhandlerci[0]{\vhandler{
	\textit{return} \; x \rightarrow \comp,
	\inst_1\#\eop_1(x ; k) \rightarrow \comp,
	...,
	\inst_n\#\eop_n(x ; k) \rightarrow \comp
}}

% computations
\newcommand\comp[0]{c}
\newcommand\cif[3]{\textit{if} \; #1 \; \textit{then} \; #2 \; \textit{else} \; #3}
\newcommand\creturn[1]{\textit{return} \; #1}
\newcommand\capp[2]{#1 \; #2}
\newcommand\cop[4]{#1(#2 ; \lambda #3 . #4)}
\newcommand\copi[5]{#1 \# #2(#3 ; \lambda #4 . #5)}
\newcommand\cnew[1]{\textit{new} \; #1}
\newcommand\cdo[3]{#1 \leftarrow #2 ; #3}
\newcommand\chandle[2]{\textit{with} \; #1 \; \textit{handle} \; #2}

\lstset{
	frame = single,
	basicstyle=\ttfamily\footnotesize,
	breaklines=true
}

\title{Thesis report}
\author{Albert ten Napel}
\date{}

\begin{document}
\maketitle

\section{Simply typed lambda calculus}

As the core of the calculi that follow we have chosen the fine-grain call-by-value\cite{finegrain} variant of the simply typed lambda calculus (STLC-fg). Terms are divided in values and computations, which allows the system to be extended to have effects more easily, since values never have effects but computations do.

\begin{align*}
	\type \Coloneqq 	& 							\tag{types} \\
				& \tunit						\tag{unit type} \\
				& \tarr{\type}{\type}				\tag{type of functions} \\
	\val \Coloneqq	&							\tag{values} \\
				& x, y, z, k						\tag{variables} \\
				& \vunit						\tag{unit value} \\
				& \vabs{x}{\comp}					\tag{abstraction} \\
	\comp \Coloneqq	&							\tag{computations} \\
				& \creturn{\val}					\tag{return value as computation} \\
				& \capp{\val}{\val}					\tag{application} \\
				& \cdo{x}{\comp}{\comp}				\tag{sequencing} \\
\end{align*}

\newpage
For the typing rules there are two judgements,
$\Gamma \vdash \val : \type$ for assigning types to values and $\Gamma \vdash \comp : \type$ for assigning types to computations.

\begin{minipage}{0.33\textwidth}
	\[\frac{
	}{
		\Gamma, x : \type \vdash x : \type
	}\]
\end{minipage}
\begin{minipage}{0.33\textwidth}
	\[\frac{
	}{
		\Gamma \vdash \vunit : \tunit
	}\]
\end{minipage}
\begin{minipage}{0.33\textwidth}
\[\frac{
	\begin{array}{l}
	\Gamma, x : \type_1 \vdash \comp : \type_2
	\end{array}
}{
	\Gamma \vdash \vabs{x}{\comp} : \tarr{\type_1}{\type_2}
}\]
\end{minipage}

\vspace{10pt}
\begin{minipage}{0.33\textwidth}
\[\frac{
	\begin{array}{l}
	\Gamma \vdash \val : \type
	\end{array}
}{
	\Gamma \vdash \creturn{\val} : \type
}\]
\end{minipage}
\begin{minipage}{0.33\textwidth}
\[\frac{
	\begin{array}{l}
	\Gamma \vdash \val_1 : \tarr{\type_1}{\type_2} \\
	\Gamma \vdash \val_2 : \type_1
	\end{array}
}{
	\Gamma \vdash \capp{\val_1}{\val_2} : \type_2
}\]
\end{minipage}
\begin{minipage}{0.33\textwidth}
\[\frac{
	\begin{array}{l}
	\Gamma \vdash \comp_1 : \type_1 \\
	\Gamma , x : \type_1 \vdash \comp_2 : \type_2 \\
	\end{array}
}{
	\Gamma \vdash (\cdo{x}{\comp_1}{\comp_2}) : \type_2
}\]
\end{minipage}

We define the relation $\rightsquigarrow$ for the small-step operational semantics. \\
\begin{minipage}{0.5\textwidth}
\[\frac{
}{
	\capp{(\vabs{x}{\comp})}{\val} \rightsquigarrow \comp[\val / x]
}\]
\end{minipage}
\begin{minipage}{0.5\textwidth}
\[\frac{
	\comp_1 \rightsquigarrow \comp_1'
}{
	(\cdo{x}{\comp_1}{\comp_2}) \rightsquigarrow (\cdo{x}{\comp_1'}{\comp_2})
}\]
\end{minipage}
\[\frac{
}{
	(\cdo{x}{\creturn{\val}}{\comp}) \rightsquigarrow \comp[\val / x]
}\]

\inputminted{haskell}{code1.txt}

\newpage
\section{STLC-fg with effects}
We now add effects to STLC-fg. Computations can use effects, these will be annotated in function types.
We assume there is a predefined set of effects $\steffs := \{ \eff_1, ..., \eff_n \}$, where $\eff$ is a single effect name and $\effs$ is some subset of $\steffs$.
In a real programming language these effects would include IO, non-determinism, concurrency, mutable state and so on.\\
In the syntax we only change the type of functions to include the effects that will be performed when applying the function.
\begin{align*}
	\type \Coloneqq 	& ...							\tag{types} \\
				& \tarre{\type}{\effs}{\type}			\tag{type of functions} \\
\end{align*}

In the typing judgment of computations we now also capture the effects that are performed: $\Gamma \vdash \comp : \type \; ; \effs$. \\
For the value typing rules only the abstraction rule changes:
\[\frac{
	\begin{array}{l}
	\Gamma, x : \type_1 \vdash \comp : \type_2 \; ; \effs
	\end{array}
}{
	\Gamma \vdash \vabs{x}{\comp} : \tarre{\type_1}{\effs}{\type_2}
}\]
In the computation typing rules we now have to pass through the effects:
\begin{minipage}{0.33\textwidth}
\[\frac{
	\begin{array}{l}
	\Gamma \vdash \val : \type
	\end{array}
}{
	\Gamma \vdash \creturn{\val} : \type \; ; \effs
}\]
\end{minipage}
\begin{minipage}{0.33\textwidth}
\[\frac{
	\begin{array}{l}
	\Gamma \vdash \val_1 : \tarre{\type_1}{\effs}{\type_2} \\
	\Gamma \vdash \val_2 : \type_1
	\end{array}
}{
	\Gamma \vdash \capp{\val_1}{\val_2} : \type_2 \; ; \effs
}\]
\end{minipage}
\begin{minipage}{0.33\textwidth}
\[\frac{
	\begin{array}{l}
	\Gamma \vdash \comp_1 : \type_1 \; ; \effs \\
	\Gamma , x : \type_1 \vdash \comp_2 : \type_2 \; ; \effs \\
	\end{array}
}{
	\Gamma \vdash (\cdo{x}{\comp_1}{\comp_2}) : \type_2 \; ; \effs
}\]
\end{minipage}

The semantics do not need to be changed.
\newpage
\inputminted{haskell}{code2.txt}

\newpage
\section{Algebraic effects}

We extend the calculus with basic algebraic effects. For every effect $\eff$ we now have a set of operations $\eops{\eff}$, where $\eop$ is a single operation and $\eopa$ is some set of operations (possible from different effects). Each operation $\eop$ has a parameter type $\type^0_\eop$ and a return type $\type^1_\eop$. We extend and update the syntax of the previous calculus as follows:

\begin{align*}
	\type \Coloneqq 	& ...							\tag{types} \\
				& \tarr{\type}{\eopa}{\type}			\tag{type of functions}\\
				& \thandler{\eopa}{\type}{\eopa}{\type}	\tag{type of handlers} \\
	\val \Coloneqq	& ...							\tag{values} \\
				& \vhandlerc						\tag{handler} \\
	\comp \Coloneqq	& ...							\tag{computations} \\
				& \cop{\eop}{\val}{x}{\comp}			\tag{operation call} \\
				& \chandle{\val}{\comp}				\tag{handle computation} \\
\end{align*}

In the function type we now have a set of operations instead of effects, both is possible but having operations there simplifies the typing rules. \\
Note that the operation call also packages a value and a continuation inside of it, having the continuation makes the semantics easier.
We can get back the simpler operation calls such as seen in Eff and Koka by defining $\eop := \vabs{x}{\cop{\eop}{x}{y}{\creturn{y}}}$ (Pretnar calls these Generic Effects in \cite{pretnar}).

\subsection{Typing rules}
In the typing judgment of computations we now capture the operations that are performed: $\Gamma \vdash \comp : \type \; ; \eopa$. \\
The typing rules for the values of the previous calculus stay the same.
For the handler we check that both the return case and the operation cases agree on the effects.
A handler is allowed to have more effects in its type then it handles, these effects will simply
remain unhandled and they will appear in both the input and output effect sets.
The input effect set must atleast contain all effects that the operations belong to and the output effect set
must agree on the unhandled effects of the input set.
\[\frac{
	\begin{array}{l}
	\Gamma , x_r : \type_1 \vdash \comp_r : \type_2 \; ; \eopa_2 \\
	\Gamma , x_i : \type^0_{\eop_i} , k_i : \tarre{\type^1_{\eop_i}}{\eopa_2}{\type_2} \vdash \comp_i : \type_2 \; ; \eopa_2 \\
	\eopa_1 \setminus \eopa_i \subseteq \eopa_2 \\
	\end{array}
}{
	\begin{array}{l}
	\Gamma \vdash \vhandler{
		\textit{return} \; x_r \rightarrow \comp_r,
		\eop_1 ( x_1 ; k_1 ) \rightarrow \comp_1,
		...,
		\eop_n ( x_n ; k_n ) \rightarrow \comp_n
	} \\ \indent : \thandler{\eopa_1}{\type_1}{\eopa_2}{\type_2}
	\end{array}
}\]

The typing rules of the computations of the previous calculus stay the same. For operation calls we have to check that the effect that belongs to the operation is contained in the resulting effect set. \\
\begin{minipage}{0.5\textwidth}
\[\frac{
	\begin{array}{l}
	\Gamma \vdash \val : \type^0_\eop \\
	\Gamma , x : \type^1_\eop \vdash \comp : \type \; ; \eopa \\
	\eop \in \eopa \\
	\end{array}
}{
	\Gamma \vdash \cop{\eop}{\val}{x}{\comp} : \type \; ; \eopa
}\]
\end{minipage}
\begin{minipage}{0.5\textwidth}
\[\frac{
	\begin{array}{l}
	\Gamma \vdash \val : \thandler{\eopa_1}{\type_1}{\eopa_2}{\type_2} \\
	\Gamma \vdash \comp : \type_1 \; ; \eopa_1 \\
	\end{array}
}{
	\Gamma \vdash \chandle{\val}{\comp} : \type_2 \; ; \eopa_2
}\]
\end{minipage}

\subsection{Semantics}

Following are the small-step operational semantics of the calculus taken from \cite{pretnar}.
The rule for abstractions stays the same.
With the computations we can always either get to $\creturn{\val}$ or $\cop{\eop}{\val}{x}{\comp}$ by floating out the operation call.
This makes the semantics of the handle computation easier since we only have to consider the cases of return and operation calls.

\begin{minipage}{0.5\textwidth}
\[\frac{
	\comp_1 \rightsquigarrow \comp_1'
}{
	(\cdo{x}{\comp_1}{\comp_2}) \rightsquigarrow (\cdo{x}{\comp_1'}{\comp_2})
}\]
\end{minipage}
\begin{minipage}{0.5\textwidth}
\[\frac{
}{
	(\cdo{x}{\creturn{\val}}{\comp}) \rightsquigarrow \comp[\val / x]
}\]
\end{minipage}
\[\frac{
}{
	(\cdo{x}{\cop{\eop}{\val}{y}{\comp_1}}{\comp_2}) \rightsquigarrow \cop{\eop}{\val}{y}{(\cdo{x}{\comp_1}{\comp_2})}
}\]

\vspace{20pt}
$h := \vhandler{
		\textit{return} \; x_r \rightarrow \comp_r,
		\eop_1 (  x_1 ; k_1 ) \rightarrow \comp_1,
		...,
		\eop_n ( x_n ; k_n ) \rightarrow \comp_n
	}$, in the following rules:\\
	
\begin{minipage}{0.5\textwidth}
\[\frac{
	\comp \rightsquigarrow \comp'
}{
	\chandle{h}{\comp} \rightsquigarrow \chandle{h}{\comp'}
}\]
\end{minipage}
\begin{minipage}{0.5\textwidth}
\[\frac{
}{
	\chandle{h}{(\creturn{\val})} \rightsquigarrow \comp_r[\val / x_r]
}\]
\end{minipage}
\vspace{20pt}
\[\frac{
	\eop_i \in \{ \eop_1, ..., \eop_n \}
}{
	\chandle{h}{\cop{\eop_i}{\val}{x}{\comp}} \rightsquigarrow \comp_i[\val / x_i, (\vabs{x}{\chandle{h}{\comp}}) / k_i]
}\]

\[\frac{
	\eop \notin \{ \eop_1, ..., \eop_n \}
}{
	\chandle{h}{\cop{\eop}{\val}{x}{\comp}} \rightsquigarrow \cop{\eop}{\val}{x}{\chandle{h}{\comp}}
}\]

\subsection{Examples}
\inputminted{haskell}{code3.txt}

\newpage
\section{Algebraic effects with static instances}

Static instances allow multiple ``versions'' of some effect to be used. This allows you to handle the same effect in multiple ways in the same program. We assume there is some statically known set of instances $\allinsts$, for each effect $\eff$ there is a set of instances $\insts{\eff} \subseteq \allinsts$. A single instance is denoted as $\inst$ and a set of instances as $\instss$. We denote the pair of an instance and an operation as $\inst\#\eop$ and the set of pairs $\inst\#\eop^*$. We call operations on instances, using $\inst\#\eop$. We update the syntax of the previous calculus as follows:

\begin{align*}
	\type \Coloneqq 	& ...									\tag{types} \\
				& \tarr{\type}{\inst\#\eop^*}{\type}				\tag{type of functions}\\
				& \thandler{\inst\#\eop^*}{\type}{\inst\#\eop^*}{\type}	\tag{type of handlers} \\
	\val \Coloneqq	& ...									\tag{values} \\
				& \vhandlerci								\tag{handler} \\
	\comp \Coloneqq	& ...									\tag{computations} \\
				& \copi{\inst}{\eop}{\val}{x}{\comp}				\tag{operation call} \\
\end{align*}

\subsection{Typing rules}
In the typing judgment of computations we also need to note the instances that are used: $\Gamma \vdash \comp : \type \; ; \inst\#\eop^*$. 
\[\frac{
	\begin{array}{l}
	\Gamma , x_r : \type_1 \vdash \comp_r : \type_2 \; ; \inst\#\eop^*_2 \\
	\Gamma , x_i : \type^0_{\eop_i} , k_i : \tarre{\type^1_{\eop_i}}{\inst\#\eop^*_1}{\type_2} \vdash \comp_i : \type_2 \; ; \inst\#\eop^*_2 \\
	\inst\#\eop^*_1 \setminus \inst\#\eop^*_i \subseteq \inst\#\eop^*_2 \\
	\end{array}
}{
	\begin{array}{l}
	\Gamma \vdash \vhandler{
		\textit{return} \; x_r \rightarrow \comp_r,
		\inst_1\#\eop_1 ( x_1 ; k_1 ) \rightarrow \comp_1,
		...,
		\inst_n\#\eop_n ( x_n ; k_n ) \rightarrow \comp_n
	} \\ \indent : \thandler{\inst\#\eop^*_1}{\type_1}{\inst\#\eop^*_2}{\type_2}
	\end{array}
}\]
\begin{minipage}{0.5\textwidth}
\[\frac{
	\begin{array}{l}
	\Gamma \vdash \val : \type^0_\eop \\
	\Gamma , x : \type^1_\eop \vdash \comp : \type \; ; \inst\#\eop^* \\
	\inst\#\eop \in \inst\#\eop^* \\
	\end{array}
}{
	\Gamma \vdash \copi{\inst}{\eop}{\val}{x}{\comp} : \type \; ; \inst\#\eop^*
}\]
\end{minipage}
\begin{minipage}{0.5\textwidth}
\[\frac{
	\begin{array}{l}
	\Gamma \vdash \val : \thandler{\inst\#\eop^*_1}{\type_1}{\inst\#\eop^*_2}{\type_2} \\
	\Gamma \vdash \comp : \type_1 \; ; \inst\#\eop^*_1 \\
	\end{array}
}{
	\Gamma \vdash \chandle{\val}{\comp} : \type_2 \; ; \inst\#\eop^*_2
}\]
\end{minipage}

\subsection{Semantics}

The small-step semantics stay very similar to the semantics of the basic algebraic effects, we just have to add the instances to the operation calls.

\[\frac{
}{
	(\cdo{x}{\copi{\inst}{\eop}{\val}{y}{\comp_1}}{\comp_2}) \rightsquigarrow \copi{\inst}{\eop}{\val}{y}{(\cdo{x}{\comp_1}{\comp_2})}
}\]

\vspace{20pt}
$h := \vhandler{
		\textit{return} \; x_r \rightarrow \comp_r,
		\inst_1\#\eop_1 (  x_1 ; k_1 ) \rightarrow \comp_1,
		...,
		\inst_n\#\eop_n ( x_n ; k_n ) \rightarrow \comp_n
	}$, in the following rules:
	
\[\frac{
	\inst\#\eop_i \in \{ \inst_1\#\eop_1, ..., \inst_n\#\eop_n \}
}{
	\chandle{h}{\copi{\inst}{\eop_i}{\val}{x}{\comp}} \rightsquigarrow \comp_i[\val / x_i, (\vabs{x}{\chandle{h}{\comp}}) / k_i]
}\]

\[\frac{
	\inst\#\eop \notin \{ \inst_1\#\eop_1, ..., \inst_n\#\eop_n \}
}{
	\chandle{h}{\copi{\inst}{\eop}{\val}{x}{\comp}} \rightsquigarrow \copi{\inst}{\eop}{\val}{x}{\chandle{h}{\comp}}
}\]

\subsection{Examples}
\inputminted{haskell}{code4.txt}

\newpage
\section{Algebraic effects with first-class static instances}
\subsection{Typing rules}
\subsection{Semantics}
\subsection{Examples}
\inputminted{haskell}{code5.txt}

\newpage
\section{Algebraic effects with dynamic instances}
\subsection{Typing rules}
\subsection{Semantics}
\subsection{Examples}

\begin{thebibliography}{9}
\bibitem{finegrain}
Levy, PaulBlain, John Power, and Hayo Thielecke. "Modelling environments in call-by-value programming languages." Information and computation 185.2 (2003): 182-210.
\bibitem{pretnar}
Pretnar, Matija. "An introduction to algebraic effects and handlers. invited tutorial paper." Electronic Notes in Theoretical Computer Science 319 (2015): 19-35.
\bibitem{effectsystem}
Bauer, Andrej, and Matija Pretnar. "An effect system for algebraic effects and handlers." International Conference on Algebra and Coalgebra in Computer Science. Springer, Berlin, Heidelberg, 2013.
\end{thebibliography}

\end{document}