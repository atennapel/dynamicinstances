\iffalse
\begin{enumerate}
\item Side-effects are omnipresent
\item Give examples
\item Make programs hard to understand, test, debug, hard to compile
\item many parts are pure
\item give benefits for pure parts
\item we want presice control over pure/non-pure parts
\item want to keep track of effects
\item want to use certain effects locally and encapsulated
\end{enumerate}

\textbf{Explain side-effects generally}
- side-effects
- pure, non-pure
- pros, cons
\textbf{Desires for a language}
- factor non-pure from pure
- precise control
- track effects
- encapsulation
\textbf{Algebraic effects}
- brief explanation
- pros
- cons
\textbf{Our system}
- why
\fi

Side-effects are generally used everywhere in programming.
Examples include mutable state, exceptions, nondeterminism, and user input.
Any function that includes such effects is called non-pure, while functions whose only effect is computing a result are called pure.
Side-effects often make functions hard to understand, test and debug.
This is because they rely on global state of some form, meaning that every invocation of the function may have different results.
Furthermore side-effectful programs can also be difficult to optimize, since the compiler does not have much freedom in rearranging parts of the program.
Pure functions on the other hand do not rely on any global state and thus can be reasoned about in isolation of the rest of the program.
Every time a pure function is called with the same input, it will return the same output.
This means those functions are easier to understand, test and debug.
\\\\
Even though side-effects are often an essential part of programs, many parts can still be written purely.
By factoring out the pure parts from the non-pure parts, we can still gain the benefits of pure functions for many parts of our programs.
Therefore it is desirable to a language to have precise control over which parts are pure and which are non-pure.
In addition it would also be beneficial to be able to keep track of which effects exactly are used by which function.
This gives an insight to the programmer to what a function may do when called.
Some effects can also be encapsulated, this means that although a function may use side-effects, these uses are not observational to the outside world.
We are thus unable to distinguise between a pure function and a function which locally uses encapsulated side-effects.
For example a function implementing the factorial of a number could do this functionally using recursion or imperatively using local mutable variables, where the variables do not leak outside of the function.
Both these approaches appear to the outside world as pure, since the effect of using local mutable variables in the imperative approach does not leak outside of the function.
It is therefore also desirable for a language to allow certain side-effect such as local mutable state to be encapsulated.
Lastly we would like non-pure functions to still compose as easily as pure functions do.
\\\\
Algebraic effects and handlers is an approach to programming with side-effects that has many of the desirable properties previously described.
Each effect is defined as a set of operations, for example nondeterminism can be represented by an operation which takes to values and chooses one.
Similarly state can be defined as two operations, get and put, where get is meant to return the current value of the state and put is meant to change this value.
These operations can then be called in programs, where each operation call can be seen as a branch in a computation tree.
Handlers take a program that calls operations and for each operation call defines how to proceed.
Most side-effects that are used in programming can be defined in this system.
Algebraic effects provide a way to factor out the pure parts, the operation calls, from the non-pure parts, the handlers.
Furthermore functions using different algebraic effects compose just as easily as pure functions do.
\\\\
While algebraic effects and handlers have many of the desirable properties we would like, it is unable to express mutable local variables.
In this thesis we define a calculus based on algebraic effects and handlers which allows for the definition of side-effects such as local references, local exceptions, and the dynamic opening of channels.
This system gives full control of which parts of a program are pure and non-pure.
Functions also compose easily, irrelevant of which side-effects they use.
Using a type-and-effect system we every function keeps track of which effects it may use.
We also statically ensure that side-effects are encapsulated.
We give examples of programs using these side-effects in our system and show how to implement local mutable references in this system.
We give a formal description of the syntax, typing rules and semantics of the system.

\section{Contributions}
\begin{itemize}

\item \textbf{We define a calculus based on algebraic effects and handlers which allows for the definition of local mutable state.}
We define a type-and-effect system which ensures that references do not escape their scope.
We also define a small-step operation semantics.

\item \textbf{We show how to implement state threads in this system.}
Using our sytem we implement state threads similar to a monomorphic version Haskell's ST-monad.
We show that references cannot escape their scope and that we cannot use a variable from one state thread in another state thread.

\end{itemize}

\section{Thesis structure}
