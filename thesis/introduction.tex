In this thesis we will devise a type and effect systems that can type some programs that use dynamic instances for algebraic effects and handlers.

\section{Problem statement}
% oxford comma before and (,and)

% what are effects and what problems occur
% first state problem in wider sense and state examples
% define before use, examples before use
% first: why algebraic effects

% current state of the art
% why is not enough
% what we have done


% people that know SOME functional programming
% no theoretical background

% why keep track of effects
% mention guarantees given by type system (safety guarantees)

% standard algebraic effects?
% effect handlers

% dynamically allocated cells
% explain ``state threads''
In the real world programs are effectful. They have to interact with the outside world, manage state, throw exceptions, and more.
Purely functional programming is often said to be easier to reason about than imperative programming.
This is partly because in purely functional programming effects are usually explicit.
Many functional languages use monads to encapsulate effects.
Monads make explicit what effects functions use and aid in reasoning.
Unfortunately monads are hard to compose, requiring monad transformer stacks that are hard to understand and manage.
\\
Algebraic effects and handlers is an alternative approach to effectful programming.
Using algebraic effects effects one can easily implement effects such as non-determinism, mutable state, exceptions, backtracking, and cooperative multi-threading.
These effects can be composed and ordered in any way without any extra effort required from the user.
\\
Not all effectful programs can be easily expressed using algebraic effects.
Handlers will handle all operations of a certain effects making effects such as state threads using multiple mutable references impossible to define.
These so-called dynamic effects can be implemented by adding dynamic instances to the system.
In such a system we can dynamically create instances of effects, distinct from other instances.
Handlers then only handle a specific instance, giving fine-grained control over effects.
\\
There have been multiple type-and-effect systems proposed for algebraic effects and handler but none of them account for dynamic instances. Programming with them can still result in runtime crashes because of unhandled operations.
\\
We define a type-and-effect system for algebraic effects and handlers in the presence of dynamic instances.
The system is sound with respect to progress and preservation and will ensure that a typed program with an empty effect set will have no unhandled operations.
We show that functional state threads can be expressed in this system.
We also formalize the system in Coq and prove type soundness.

\section{Proposed solution}
\begin{itemize}

% type-safe preferred, type-sound
\item \textbf{We define a type-safe type-and-effect system for algebraic effects and handlers in the presence of dynamic instances.}
% The system is sound with respect to progress and preservation and will ensure that a typed program with an empty effect set will have no unhandled operations.

% instead, core system is a principled extension, so easy to proof results and compose with other features
% and easy for users to understand. 
Furthermore the system is a simple extension of a system with algebraic effects without instances.

\item \textbf{We show how to implement state threads in this system.}
Using our sytem we implement state threads similar to a monomorphic version Haskell's ST-monad.
We show that references cannot escape their scope and that we cannot use a variable from one state thread in another state thread.

\item \textbf{We formally prove type-soundness of the system.}
We prove that typed programs with empty effects sets will not have unhandled operations and so do not get stuck.
We explain what techniques we used to implement the proofs.

%\item \textbf{We formalize the system in Coq.}
% explain why mechanization.
% mechanized, lists parts (typing rules, semantics, proofs)

\end{itemize}

\section{Thesis structure}
