\iffalse
\begin{enumerate}
\item Side-effects are omnipresent
\item Give examples
\item Make programs hard to understand, test, debug, hard to compile
\item many parts are pure
\item give benefits for pure parts
\item we want presice control over pure/non-pure parts
\item want to keep track of effects
\item want to use certain effects locally and encapsulated
\end{enumerate}

\textbf{Explain side-effects generally}
- side-effects
- pure, non-pure
- pros, cons
\textbf{Desires for a language}
- factor non-pure from pure
- precise control
- track effects
- encapsulation
\textbf{Algebraic effects}
- brief explanation
- pros
- cons
\textbf{Our system}
- why
\fi

\chapter{\label{chap:intro}Introduction}

Side-effects are ubiquitous in programming.
Examples include mutable state, exceptions, nondeterminism, and user input.
Side-effects often make functions hard to understand, test and debug.
This is because every invocation of the same function with the same arguments may yield different results.
Furthermore side-effectful programs can also be difficult to optimize, since the compiler does not have much freedom in rearranging parts of the program.
\\\\
Any function that includes such side-effects is called \emph{impure}, while functions whose only effect is computing a result are called \emph{pure}.
Pure functions do not rely on any global state and thus can be reasoned about in isolation of the rest of the program.
Every time a pure function is called with the same input, it will return the same output.
This means those functions are easier to understand, test, and debug.
\\\\
There has been a lot of work on programming languages that allow more control over the pure and impure parts of a program.
Examples include Haskell~\autocite{haskell}, Eff~\autocite{eff1}, Koka~\autocite{koka2}, and Links~\autocite{links}.
These languages, in one way or another, give the programmer more control over which parts of their program are pure and which parts are impure.
By factoring out the pure parts from the impure parts, we can still gain the benefits of pure functions for many parts of our programs.
In addition these languages allow one to keep track of which effects exactly are used by which function.
They also allow some side-effects to be encapsulated, meaning that the use of a particular side-effect can be completely hidden such that the function still appears to be pure to the outside world.
\\\\
Using type systems we can statically rule out programs that may lead to runtime errors.
Type systems can also play an essential role in enforcing the distinction between pure and impure code.
By extending type systems to also show which effects a function may use, we can statically enforce which functions are pure and which are not.
This gives insight to the user to what a function may do when called, and also allows a compiler to do more interesting optimizations.
For example pure function calls may be reordered in any way that the compiler sees fit, while impure function calls may not, since the effects may interact.
These \emph{effect systems} can have different levels of granularity.
For example one system could only keep track of a single bit per function, whether the function is impure or not.
More fine-grained systems are also possible, where each function is annotated with a set of effects that is used, where the set of possible effects is defined by the language.
For example in Koka~\autocite{koka2} a function which prints something to the console may be given the type:
\begin{minted}[tabsize=2]{haskell}
string -> <console> () 
\end{minted}
Where \textsc{console} shows the use of the console.
\\\\
Algebraic effects and handlers~\autocite{algeff} are an approach to programming with side-effects that has many of the desirable properties previously described.
Algebraic effects provide a way to factor out the pure parts from the impure parts.
Users can define effects and easily use them in functions, with different effects composing without any extra effort.
Each effect in a system with algebraic effects is defined as a set of operations.
For example nondeterminism can be represented by an operation which takes to values and chooses one.
Similarly, state can be defined as two operations, get and put, where get is meant to return the current value of the state and put is meant to change this value.
These operations can then be called anywhere in a function.
Handlers take a program that calls operations and for each operation call defines how to proceed.
For example the following piece of (pseudo)code defines an effect called \textit{State} which simulates a single mutable state cell:
\begin{minted}[tabsize=2]{haskell}
effect State {
	get : () -> Int
	put : Int -> ()
}

postInc : Int!{State}
postInc =
	x <- get ();
	put (x + 1);
	return x
	
handlePostInc : Int
handlePostInc =
	f <- handle (postInc) {
		get _ k -> \st -> k st st
		put newst k -> \st -> k () newst
		return x -> \st -> return x
	};
	f 42
\end{minted}
The function \mintinline{haskell}{postInc} increments the current value in the state cell and returns the previous value.
We then handle the calls to get and put of \mintinline{haskell}{postInc} in \mintinline{haskell}{handlePostInc} using a handler.
The handler transforms get and put calls to a function expecting the current state.
We name this function \mintinline{haskell}{f} and pass it 42 as the initial state.

\paragraph{The problem}
While algebraic effects and handlers have many of the desirable properties we would like, they are is unable to express multiple mutable references.
Mutable references have interesting applications such as meta variables in unification algorithms and typed logic variables~\autocite{typedlogic}.
In the previous example it can be seen that \mintinline{haskell}{postInc} does not refer to any specific reference, but instead can only manipulate one ambient reference using the get and put operations.
Dynamic instances were introduced by the Eff programming language~\autocite{eff1} to solve this problem.
With dynamic instances multiple different instances of the same effect can be dynamically created.
Using this multiple mutable references can be implemented by dynamically creating instances of the \mintinline{haskell}{State} effect (we give an example in \cref{sec:background-dynamicinst}).
Unfortunately theses dynamic instances can escape the scope of their handler.
Any operation called on one of these escaped instances will result in a runtime error, since this operation call will be unhandled.
Eff also introduces \emph{resources}, these are effect instances with a globally scoped handler associated.
Because the handler is globally scoped the instance can never escape its scope and any operation call will always be handled.
Unfortunately there is no type system given for dynamic instances, so we have no static guarantees that there will be no unhandled operation calls.
\\\\
In Haskell the the so-called ``ST monad''~\autocite{runst} can be used to safely manipulate multiple references in such a way that stateful computations can be encapsulated and that the references are not leaked outside of the function.
Mutable references can be dynamically created and manipulated.
Computations using these references can be made pure by passing them to a function called \mintinline{haskell}{runST}.
This function will statically ensure that no references will escape their scope and that the mutation effects are encapsulated.
\\\\
In this thesis we define a new language named \lang{} based on algebraic effects and handlers which allows for the definition of effects such as the dynamic creation of mutable references, and the opening of files and channels.
Using this system we can implement a system similar to the ST monad from Haskell.
We introduce a notion of effect scopes, which encapsulates the creation and usage of dynamically created effect instances.
Each function is statically annotated with the effect scopes that are used in the function.
Using the effect scopes we statically ensure that effects are encapsulated.
We give examples of programs using these side-effects in \lang{} and show how to implement local mutable references.
We give a formal description of the syntax, typing rules and semantics of a core calculus for \lang{}.
\\\\
Proving type safety for \lang{} turns out to be more difficult than anticipated.
It is common to prove type safety by first proving a type preservation lemma.
Type preservation states that if a term is well-typed and if we take a step (using the semantics) then the resulting term is also well-typed (with the same type as before).
Our language introduces some intermediate forms used by the small-step operational semantics.
These intermediate forms are introduced by the semantics during the process of evaluation and do not occur in the source language.
In order to prove the preservation lemma one also has to give typing rules for these intermediate forms.
We will call these the \emph{dynamic} typing rules.
The difficulty in coming up with dynamic typing rules is that effect instances can still escape their effect scope if they are not used and not returned from a function.
The dynamic typing rules have to give these escaped instance a valid type, but since they have escaped their effect scope this is tricky because the type of an instance depends on the effect scope, which it has escaped from.
The escaped instances do however have no computational effect and so we conjecture that type safety still holds.
Type safety might still be provable in other ways, given correct dynamic typing rules.
In this thesis we will show the problems with proving type preservation for \lang{} and we will discuss other approaches that may allow us to prove type safety.

\section*{Contributions}
\begin{itemize}

\item \textbf{Language.}
We define a language named \lang{} based on algebraic effects and handlers that can handle a form of dynamic effect instances.

\item \textbf{Mutable references.}
We give examples in \lang{} that would be difficult or impossible to express with ordinary algebraic effects.

\item \textbf{Operational semantics and type system.}
We define a core calculus of \lang{} together with a small-step operational semantics and a type system.

\item \textbf{Research on type safety for dynamic instances.}
We show the problems we encountered when trying to prove type safety for \lang{}.
We discuss the other approaches that may allow us to prove type safety.

\item \textbf{Formalizations.}
We have formalized algebraic effects and handlers with and without static instances and have proven type soundness in Coq\footnote{https://github.com/atennapel/dynamicinstances}.

\end{itemize}

\section*{Thesis structure}
The thesis is structured as follows.
\Cref{chap:algintro} gives an introduction to ordinary algebraic effects and handlers, and static and dynamic instances.
\Cref{chap:langintro} gives an introduction to our new language \lang{}.
\Cref{chap:algtheory} gives formal definitions of systems with ordinary algebraic effects and handlers, and static instances.
\Cref{chap:langtheory} gives a formal account of \lang{}, we also discuss the problem with proving type safety.
\Cref{chap:related} discusses related work.
\Cref{chap:conclusion} concludes the thesis and discusses future work.
