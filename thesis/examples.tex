\iffalse
In this chapter we will give an introduction to programming with X.
X is a modification of the algebraic effects system described in Chapter~2.1, extended with a restricted form of the dynamic instances of Chapter~2.3.
We introduce the novel concepts of X.
When create a new instance we have to give handler, this ensures that the instance is always associated with a handler.
The $\mathsf{runscope}$ construct can handle multiple effects at the same time.
These effects are collected under a notion of \emph{effect scope}, which is used to group instances together.
When runscope is called one such effect scope is handled.
We introduce \emph{effect scope polymorphism} together with \emph{effect scope abstraction} and {effect scope application}.
These constructs allows us to write functions which are polymorphic over effect scopes.
\fi

In Chapter~2.3 we saw how dynamic instances allow us to implement mutable references in a system with algebraic effects.
This system is untyped however, meaning that you can get runtime errors if an operation is unhandled.
This can happen if an operation is called on a dynamic instance outside of a handler.
In this chapter we introduce X, which combines algebraic effects and dynamic instances while still remaining typeable.
In order to achieve this we introduce the notion of an \emph{effect scope}.
An effect scope groups together instances.
When creating an instance we give an effect scope to create the instance in.
Every instance belongs to a specific effect scope.
Different from the system with dynamic instances from Chapter~2.3, we always have to specify a handler when creating an instance, similar to resources in Eff.
Specifying a handler at the moment of creating an instance ensures that each instance always has a handler associated with it.
Performing effects is done with a new $\mathsf{runscope}$ construct, similar to how the $\mathsf{handle}$ construct performed effects in Chapter~2.
$\mathsf{runscope}$ creates a fresh scope and makes it available for use in a given computation.
After all the instances created on this new scope will actually be instantiated and all the operations called on these instances will be handled.
In order to allow computations to be polymorphic over effect scopes we also introduce \emph{effect scope polymorpishm} together with \emph{effect scope abstraction} and \emph{effect scope application}.
\\\\
We start with explaining all the novel concepts in Section~3.1, using the example of mutable references.
% After we will show how mutable references can be expressed in X in Section~3.2.
Then we will show how mutable vectors can be defined, followed by an implementation of a list shuffling algorithm in Section~3.2.
\iffalse
We end the chapter by showing how our system allows for local effects in Section~3.3.
\fi
\\\\
We build on the language used in Chapter~2.1.
We use syntax reminiscent of Haskell with algebraic data types and pattern matching.
Type constructors and effect names are uppercase while type variables are lowercase.
% The language also has parametric polymorphism over effect scopes.
% We will always explicitly show \ident{forall} for universally quantified effect scope variables.

\section{Effects, effect scopes and instances}
\label{section:basics}

\begin{figure}
\caption{Example of all the novel constructs}
\begin{minted}[tabsize=2,frame=single,linenos]{haskell}
effect State {
	get : () -> Int
	put : Int -> ()
}

ref : forall s. Int -> (Inst s State)!{s}
ref [s] v =
	new State@s {
		get () k -> \st -> k st st
		put st' k -> \st -> k () st'
		return x -> \st -> return x
		finally f -> f v
	} as x in return x

postInc : forall s. Inst s State -> Int!{s}
postInc [s] inst =
	x <- inst#get();
	inst#put(x + 1);
	return x

result : Int!{}
result =
	runscope(s1 ->
		r1 <- ref [s1] 10;
		runscope(s2 ->
			r2 <- ref [s2] 20;
			x <- postInc [s2] r2;
			r1#put(x);
			return x);
		y <- r1#get ();
		return y) -- result is 20
\end{minted}
\label{fig:example1}
\end{figure}

In Figure~\ref{fig:example1} we give an example containing all the novel constructs of X.

\subsection{Effects}

To start off we define a \ident{State} effect specialized to \ident{Int}s.
\ident{State} is meant to represent a mutable reference to a single value of type \ident{Int}.
This definition is exactly the same as the \ident{State} effect definition in Chapter~2.1, in the basic algebraic effects system.
With the \ident{effect} keyword we declare a new effect called \ident{State} with two operations: \ident{get} and \ident{put}.
For each operation we give parameter and return types. For \ident{get} we give the unit type \ident{()} as the parameter type.
As the return type we give \ident{Int}, meaning that calling the \ident{get} operation will return a integer value.
For the \ident{put} operation it is the other way around.
As the parameter type we have \ident{Int}, meaning that \ident{put} requires an integer value when called, and the return type is \ident{()}, calling \ident{put} will give back the unit value.

\subsection{Effect instances}
In Figure~\ref{fig:example1} the function \ident{ref} creates a new \emph{effect instance} of the \ident{State} effect.
We can create a fresh instance of an effect using the \ident{new} keyword.
The construct \ident{new State@s { ... }} can be read as ``Create a fresh \ident{State} instance in the effect scope \ident{s}''.
Here we have to give a specific scope \ident{s} to create the instance on.
The instance can only be used within this given scope.
The newly-created instance is available in the body of the \ident{new} construct.
When creating an instance we have to give a \emph{handler}.
The handler specifies what should happen when the operations are called.
The handler is defined within curly braces and consists of a case for each operation of the effect, plus a \ident{return} case and a \ident{finally} case.
The handler given in the example is the same as the handler given in Chapter~2.1, implementing a single mutable reference.
In addition we also have to define a \ident{finally} case.
In there we can perform an extra computation after the handler is performed.
In the case of the handler given this is necessary because the return type of the handler is \ident{Int -> T} for some return type \ident{T}.
The handler transforms a computation in to a function expected the initial value for the mutable reference.
Using the \ident{finally} case we can call this function and get back the return value of the computation (of type \ident{T}.
In the example we simply call the function \ident{f} with the argument \ident{v} given to \ident{ref}.
\\\\
We can now understand the type of \ident{ref}:
\begin{minted}[tabsize=2]{haskell}
ref : forall s. Int -> (Inst s State)!{s}
\end{minted}
We can see from the \ident{forall s.} that \ident{ref} is polymorphic over scopes.
That means that this function works for any scope \ident{s}.
\\\\
The type variable \ident{s} here is an \emph{effect scope variable}.
An effect scope variable can be seen as the name of a collection of instances that we call an effect scope.
Such a scope can contain zero or more instances, where each instance can be of any effect.
A scope restricts instances in such a way that they cannot escape that scope and instances from one scope cannot be used in another.
This also means that we can never get a runtime error because of an unhandled operation call.
\\\\
In order to apply \ident{ref} we have to give an explicit scope \ident{s} using the syntax \ident{ref [s]}.
We call this \emph{effect scope application}.
In the definition we show that this function has a scope parameter using angle brackets \ident{[s]}.
We call this \emph{effect scope abstraction}.
The second parameter is a value of type \ident{Int} which we call \ident{v}.
This is the initial value that we want our mutable reference to have.
The return type is \ident{Inst s State}.
This is an effect instance of the \ident{State} effect in the scope \ident{s}.
From this type we can see that \ident{ref} gives back an instance on the given scope \ident{s}.
The effect annotation of \ident{ref} is \ident{!{s}}, which shows that we actually perform effects in the scope \ident{s}.
We can see from the function implementation that the only effect we perform is the creation of an instance on \ident{s}.
\\\\
Using \ident{ref} we can fully emulate multiple mutable references.
We have the added guarantee that the references will not escape their effect scope, they will not escape their corresponding \ident{runscope}.
Adding parametric polymorphism to the effects to give \ident{State t} for any type \ident{t} will enable us to emulate references of any type.
With references of different types coexisting.
This is very similar to how the ST monad works in Haskell~\cite{runst}.
Using mutable references have interesting applications such as meta variables in unification algorithms and typed logic variables~\cite{typedlogic}.
\\\\
Looking at the type of creating mutable references using the ST monad in Haskell (\ident{newSTRef}), we can see that the type of \ident{ref} is very similar
(we explicitly wrote down the quantification and specialized \ident{newSTRef} to \ident{Int}):

\begin{minted}[tabsize=2]{haskell}
ref : forall s. Int -> (Inst s State)!{s}
newSTRef :: forall s. Int -> ST s (STRef s Int)
\end{minted}

Here \ident{ST s} serves the same purpose as \ident{!{s}} in our system.
\ident{STRef s Int} is the type of a mutable reference in the ST monad in Haskell.
\ident{s} is some ``state thread'', the purpose if this type variable is to statically ensure that references do not escape their scope.
This is exactly like the \ident{s} type variable in our system, but we generalize ``state threads'' to effect scopes, where any algebraic effect may be performed.

\subsection{Using effect instances}
\label{subsection:instances}
In Figure~\ref{fig:example1} the function \ident{postInc} shows how an effect instance can actually be used:
\begin{minted}[tabsize=2]{haskell}
postInc : forall s. Inst s State -> Int!{s}
postInc [s] inst =
	x <- inst#get();
	inst#put(x + 1);
	return x
\end{minted}

We can see from the type that this function is polymorphic over some effect scope \ident{s}.
It expects some scope \ident{s} and some \ident{State} instance on \ident{s} as its arguments.
It returns an integer value and may perform some effects on \ident{s} (\ident{Int!{s}}.
The type of \ident{postInc} can be read as ``For any scope \ident{s}, given a \ident{State} instance in \ident{s}, return a value of type \ident{Int} possibly by calling operations on instances in \ident{s}''.
\\\\
Effects can be used by calling operations.
Operations are always called on an effect instance.
Without an instance we are unable to perform operations.
In the case of \ident{postInc} we get an instance as an argument to the function.
Operation can be called on an instance using the syntax \ident{instance\#operation(argument)}.
We write \ident{instance\#operation()} to mean \ident{instance\#operation(())}, when the unit value \ident{()} is given as the argument.
\ident{postInc} implements the traditional ``post increment'' operation on a mutable reference.
In the C language this is written \ident{x++} for some reference \ident{x}.
We first call the \ident{get} operation on \ident{inst}.
We get back a value of type \ident{Int}, which we name \ident{x}.
Then we call \ident{put} on \ident{inst} with the argument \ident{(x + 1)}.
Finally we return the previous value of the mutable reference \ident{x}.
\\\\
We could define functions wrapping the \ident{get} and \ident{put} operations:

\begin{minted}[tabsize=2]{haskell}
performGet : forall s. Inst s State -> Int!{s}
performGet [s] inst = inst#get()

performPut : forall s. Inst s State -> Int -> ()!{s}
performPut [s] inst v = inst#put(v)
\end{minted}

Again we can compare to the corresponding functions in the ST monad in Haskell, \ident{readSTRef} and \ident{writeSTRef}:

\begin{minted}[tabsize=2]{haskell}
readSTRef :: forall s. STRef s Int -> ST s Int
writeSTRef :: forall s. STRef s Int -> Int -> ST s ()
\end{minted}

We can see that the types are very similar, but that we generalize the ST monad to work for any algebraic effect.

\subsection{Running scopes}
The definition \ident{result} shows how the effects in a computation can be performed:

\begin{minted}[tabsize=2]{haskell}
result : Int!{}
result =
	runscope(s1 ->
		r1 <- ref [s1] 10;
		ret <- runscope(s2 ->
			r2 <- ref [s2] 20;
			x <- postInc [s2] r2;
			r1#put(x);
			return x);
		y <- r1#get ();
		return y) -- result is 20
\end{minted}

From the type we can see that \ident{result} is a computation that returns an integer value.
We can see from the effect annotation (\ident{!{}}) that \ident{result} does not have any unhandled effects.
In the future we will omit writing \ident{!{}} if a computation does not have any unhandled effects.
\\\\
The \ident{runscope(s' -> ...)} construct provides a new scope, which we named \ident{s1} and \ident{s2} in our case, which can be used in its body.
Inside \ident{runscope} we can create and use instances in this new scope.
\ident{runscope} will make sure that any instances that are created on its scope will actually be created and that any operation calls on these instances will be handled.
\\\\
In \ident{result} we use two nested scope.
First we create a scope called \ident{s1}.
On this scope we call \ident{ref} to create a mutable reference \ident{r1} with \ident{10} as its initial value.
Then we create another scope called \ident{s2}.
In \ident{s2} we create another mutable reference \ident{r2} with \ident{20} as its initial value.
We then call \ident{postInc} on \ident{r2} and store the return value in \ident{x} (\ident{20}).
Then we call \ident{r1#put(x)}, setting \ident{r1} to \indent{20}.
We then return \ident{x} as the return value of the \ident{s2} scope, storing this value in \ident{ret} in the \ident{s1} scope.
At this point the \ident{s2} scope is gone and any effects in it will be handled.
The type system will make sure that no instances created in \ident{s2} can escape \ident{s2}.
\\\\
Note that we also used \ident{r1} inside \ident{s2}.
Since \ident{r1} belongs to \indent{s1}, all the operations called on it will not be handled inside \ident{s2} but these will be \emph{forwarded} instead.
This means that these operations will remain unhandled until \ident{s2} is done.
Because of this forwarding behaviour we can combine effects from multiple scopes, giving us fine-grained control over where effects may happen.
\\\\
Continuing in \ident{result} we get the current value of \ident{r1} and return from \ident{s1}.
This value is \ident{20} which was set in scope \ident{s2}.
After this the scope \ident{s1} is done and any effect in it will be handled.
This leaves us with a computation of type \ident{Int!{}} with no remaining effects.

\section{Mutable vectors}
\begin{figure}[h]
\caption{Mutable vectors}
\begin{minted}[tabsize=2,frame=single,linenos]{haskell}
-- list of mutable references
data Vector s = VNil | VCons (Inst s State) Vector

-- get the length of a vector
vlength : forall s. Vector s -> Int
vlength VNil = 0
vlength (VCons _ tail) = 1 + (vlength tail)

-- get the value at the index given as the first argument
-- assumes the index is within range of the vector
vget : forall s. Int -> Vector s -> Int!{s}
vget [s] 0 (VCons h _) = h#get()
vget [s] n (VCons _ t) = vget [s] (n - 1) t

-- set the value at the index given as the first argument
-- to the value given as the second argument
-- assumes the index is within range of the vector
vset : forall s. Int -> Int -> Vector s -> ()!{s}
vset [s] 0 v (VCons h _) = h#put(v)
vset [s] n v (VCons _ t) = vset [s] (n - 1) v t
\end{minted}
\label{fig:vectors}
\end{figure}

In the previous section we have defined mutable references using the \ident{ref} function.
We will now build on them to define mutable vectors.
In Figure~\ref{fig:vectors} we define the \ident{Vector} datatype.
\ident{Vector} is a list of \ident{State} instances and is indexed by the scope of instances: \ident{s}.
We define three functions on \ident{Vector}: \ident{vlength}, \ident{vget}, and \ident{vset}.
With \ident{vlength} we can get the length of a vector.
With \ident{vget} we can retrieve a value from a vector by giving an index.
We assume the index is within the range of the vector.
With \ident{vset} we can set an element of a vector by giving an index and a value.
Again we assume the index is within the range of the vector.
In order to allow these functions to work for any vector we have to introduce an effect scope variable \ident{s} again.
We define both functions by recursion on the index.
\\\\
As an example application we will write a shuffling algorithm for vectors.
This simple algorithm will shuffle a vector by randomly swapping two random elements of the vector and repeating this some amount of times.
In Figure~\ref{fig:shuffle} we show the algorithm.
First we define an effect \ident{Rng} in order to abstract out the generation of random numbers.
\ident{Rng} has a single operation \ident{rand} which returns a random integer between \ident{0} and \ident{n} given an integer \ident{n}.
We define a function \ident{vlength} to get the length of the vector.
\\\\
We then define the actually shuffling function \ident{shuffleVector}.
This function takes two scope variables, \ident{s} and \ident{s'}, for the vector and \ident{Rng} instance respectively.
As arguments we take an instance of \ident{Rng}, in order to generate random numbers, an integer, for the amount of times to shuffle, and the vector we want to shuffle.
By taking a seperate scope for the \ident{Rng} instance we are more flexible when handling the computation.
We can handle the effects on the vector while leaving the \ident{Rng} effects to be handled higher up.
\\\\
\ident{shuffleVector} proceeds as follows.
If the amount of times we want to shuffle is \ident{0} we stop and return the vector.
If not then we first get the length of the vector.
Then we generate two random numbers, \ident{i} and \ident{j}, between \ident{0} and this length.
These two numbers will be the two elements we will swap.
We then get the current values at these indeces.
And we swap the values at these indeces in the vector.
We then recurse, subtracting the amount of times to shuffle by one.

\begin{figure}[H]
\caption{Vector shuffling}
\begin{minted}[tabsize=2,frame=single,linenos]{haskell}
-- random number generation effect
-- the operation `rand` gives back a random integer
-- between 0..n, where n is the argument given (exclusive)
effect Rng {
	rand : Int -> Int
}

-- shuffles a vector given an instance of Rng
-- by swapping two random elements of the vector
-- the second argument to shuffleVector is the amount of times
-- to swap elements
shuffleVector : forall s s'. Inst s' Rng -> Int
	-> Vector s -> ()!{s, s'}
shuffleVector [s] [s'] _ 0 vec = vec
shuffleVector [s] [s'] rng n vec =
	let len = vlength vec;
	i <- rng#rand(len);
	j <- rng#rand(len);
	a <- vget [s] i vec;
	b <- vget [s] j vec;
	vset [s] i b vec;
	vset [s] j a vec;
	shuffleVector [s] [s'] rng (n - 1) vec
\end{minted}
\label{fig:shuffle}
\end{figure}

Using \ident{shuffleVector} we can implement a function to shuffle a list in Figure~\ref{fig:listshuffle}.
We first define the usual \ident{List} datatype, with \ident{Nil} and \ident{Cons} cases.
Then we define two functions \ident{toVector} and \ident{toList} to convert between lists and vectors.
\ident{toVector} simply recurses on the list and creates fresh mutable references for each element of the list, initialized with the value of the element.
\ident{toList} converts a vector to a list by getting the current values of each reference in the vector.
The function \ident{shuffle} implements the actual shuffling.
It takes an effect scope, a \ident{Rng} instance \ident{rng} in this scope and a list \ident{lst}.
We first convert the list to a mutable vector.
Then we use \ident{shuffleVector} to shuffle the vector 100 times, passing \ident{rng} for generating the random numbers.
Finally we convert the vector back to a list and return this result.
We wrap this computation in \ident{runscope} to handle the effects of the mutable vector.
The use of mutable vectors is not leaked outside of the function, from the type and behaviour of \ident{shuffleVector} we are unable to find out if mutable vectors are used.
We say that the use of the \ident{State} effect is completely \emph{encapsulated}.
The type system ensures that \ident{runscope} actually does encapsulate all effects in its scope.
Note that we do not handle the scope of \ident{rng}, we leave the \ident{Rng} to be handled higher up by the caller of \ident{shuffle}.

\begin{figure}
\caption{List shuffling}
\begin{minted}[tabsize=2,frame=single,linenos]{haskell}
-- (linked) list of integer values
data List = Nil | Cons Int List

-- transform a list to a vector by replacing each value
-- in the list by a reference initialized with that value
toVector : forall s. List -> (Vector s)!{s}
toVector [s] Nil = VNil
toVector [s] (Cons h t) =
	h' <- ref [s] h;
	t' <- toVector [s] t;
	return (VCons h' t')

-- transform a vector back to a list by getting the
-- current values from the references in the vector
toList : forall s. Vector s -> List!{s}
toList [s] VNil = Nil
toList [s] (VCons h t) =
	h' <- h#get();
	t' <- toList [s] t;
	return (Const h' t')

-- shuffles a list given an instance of Rng
-- by converting it to a vector
-- and shuffling 100 times
shuffle : forall s'. Inst s' Rng -> List -> List!{s'}
shuffle [s'] rng lst =
	runscope(s ->
		let vec = toVector [s] lst;
		shuffleVector [s] [s'] rng 100 vec;
		return (toList vec))
\end{minted}
\label{fig:listshuffle}
\end{figure}

\begin{figure}
\caption{List shuffling}
\begin{minted}[tabsize=2,frame=single,linenos]{haskell}
runshuffle : List -> List
runshuffle lst =
	runscope(s ->
		seedref <- ref [s] 123456789;
		rng <- new Rng@s {
			rand n k ->
				currentseed <- seedref#get();
				let newseed = (1103515245 * currentseed + 12345) % n;
				seedref#put(newseed);
				k newseed
			return x -> return x
			finally x -> return x
		};
		shuffle [s] rng lst)
\end{minted}
\label{fig:rng}
\end{figure}

In Figure~\ref{fig:rng} we define the function \ident{runshuffle} which can shuffle a list \ident{lst}.
We use a naive implementation of random number generation using a linear congruential generator.
Inside of a new scope \ident{s} we first create a mutable reference called \ident{seedref}, which we set to a random intial seed value.
This reference will store the current state of the random number generator, which we call the seed.
We then create a fresh \ident{Rng} instance called \ident{rng}.
We implement the \ident{rand} operation by first getting the current seed value from \ident{seedref}.
Then we calculate a new seed value using arbitrarily chosen numbers, store this in \ident{seedref} and call the continuation with it.
The $\mathsf{return}$ and $\mathsf{finally}$ cases do not do anything special.
Finally we call \ident{shuffle} with our \ident{Rng} instance and \ident{lst}.
In this example we can see how we can use other effects in the handler of an instance.
The \ident{Rng} uses an instance of \ident{State} to implement the \ident{rand} operation.
Both of these effects exist and are handled in the same scope \ident{s}.
From the type of \ident{runshuffle} (\ident{List -> List}) we can see that all the effects are encapsulated and that the function is pure.
In the next Chapter we will give a formal account of the system, including a type system and semantics.

In this Chapter we have seen how to program with effect scopes.
Like the regular algebraic effects (Chapter~2.1) we can use and combine different effects simply by using their operations in a program.
What is different is that handlers are given when creating instances.
We have seen that we can abstract over and instantiate effect scopes.
Lastly we saw how effect scopes enable use to implement mutable references and vectors while still being safe.

\iffalse
\section{Local effects}
When writing higher-order functions it's important that effects used in the function definition does not interfere with the function given as an argument.
This means we want to have \emph{local effects} in those functions.
Effects which do not leak outside of the functions and which do not interfere with any other effects used.
Using effect instances we can generate effects which are local and unique.
\\\\
In Figure~\ref{fig:localeffects} we 

\begin{figure}[h]
\caption{Folding with early exit}
\begin{minted}[tabsize=2,frame=single,linenos]{haskell}
-- folding with early exit
effect Done {
	done : Bool -> ()
}

-- foldr creates a new instance of Done
-- and passes this to the reducer function
-- if done is called on the instance
-- then foldr stops and returns the value given
-- note: nested uses of foldr do not interfere because
-- new instances are created with each call
foldr : (forall s. Inst s Done -> Int -> Bool -> Bool!{s}) ->
	Bool -> List -> Bool
foldr fn initial list =
	runscope(s ->
		new Done@s {
			done v k -> return v
			return x -> return x
		} as inst in
		foldrRec [s] (fn [s] inst) initial list)

-- recursive foldr implementation
foldrRec : forall s. (Int -> Bool -> Bool!{s}) ->
	Bool -> List -> Bool!{s}
foldRec fn initial Nil = initial
foldRec fn initial (Cons h t) = fn h (foldRec fn initial t)

-- does the list have any element satifying
-- the predicate function
-- returns early if the predicate function returns True
contains : (Int -> Bool) -> List -> Bool
contains fn list =
	let result = foldr (\inst h _ ->
		if fn h then
			inst#done(True)
		else
			False
	) False list
\end{minted}
\label{fig:localeffects}
\end{figure}
\fi
