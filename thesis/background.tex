\chapter{\label{chap:algintro}An introduction to algebraic effects and handlers}

Side-effects are an essential part of a programming language. Without side-effects the program would have no way to print a result to the screen, ask for user input or change global state. We consider a function pure if it does not perform any side-effects and unpure if it does. A pure function always gives the same result for the same inputs. A pure function can be much easier to reason about than an unpure one because you know that it won't do anything else but compute, it won't have any hidden inputs or outputs. Because of this property testing pure functions is also easier, we can just give dummy inputs to the functions and observe the output. As already said programs without side-effects are useless, we would not be able to actually observe the result of a function call without side-effects such as printing to the screen. So we would the benefits of pure functions but still have side-effects. We could give up and simply add some form of side-effects to our language but that would immediately make our function impure, since any function might perform side-effects. This would make us lose the benefits of pure functions.
\\\\
Algebraic effects and handlers are a structured way to introduce side-effects to a programming language.
The basic idea is that side-effects can be described by sets of operations, called the interface of the effect.
Operations from different effects can then be called in a program.
These operations will stay abstract though, they will not actually do anything.
Instead, similar to exceptions where exceptions can be thrown and caught, operations can be ``caught'' by handlers.
Different from exceptions however the handler also has access to a continuation which can be used to continue the computation at the point where the operation was called.
\\\\
In this chapter we will introduce algebraic effects and handlers through examples.
Starting with simple algebraic effects and handlers (\cref{sec:background-algeff}).
After we will continue with static instances (\cref{sec:background-staticinst}) which allows for multiple static instances of the same effect to be used in a program.
We end with dynamic instances (\cref{sec:background-dynamicinst}) which allows for the dynamic creation of effect instances. The examples are written in a statically typed functional programming language with algebraic effects and handlers with syntax reminiscent to Haskell but semantically more similar to Koka~\autocite{koka2}.

\section{Algebraic effects and handlers}
\label{sec:background-algeff}
We will start with the familiar exceptions. We define an \ident{Exc} effect interface with a single operation \ident{throw}.

\begin{minted}[tabsize=2]{haskell}
effect Exc {
	throw : String -> Void
}
\end{minted}

For each operation in an effect interface we specify a parameter type (on the left of the arrow) and a return type (on the right of the arrow).
The parameter type is the type of a value that is given when the operation is called and that the handler also has access too.
The return type is the type of a value that has to be given to the continuation in the handler, this will be shown later.
This return value is received at the point where the operation was called.
In the case of \ident{Exc} we take \ident{String} as the parameter type, this is the error message of the exception.
An exception indicates that something went wrong and that we cannot continue in the program.
This means we do not want the program to continue at the point where the exception was thrown, which is the point where the \ident{throw} operation was called.
So we do not want to be able to call the continuation with any value.
To achieve this we specify \ident{Void} as the return type of \ident{throw}.
This is a type with no values at all, which means that the programmer will never be able to conjure up a suitable value when a value of type \ident{Void} is requested. By using \ident{Void} as the return type we can ensure that the continuation cannot be called and so that the program will not continue at the point where \ident{throw} was called. To make the code more readable we assume \ident{Void} implicitly coerces to any other type.
\\\\
We can now write functions that use the \ident{Exc} effect.
For example the following function \ident{safeDiv} which will throw an error if the right argument is $0$.
We assume here that \ident{Void} is equal to any type.

\begin{minted}[tabsize=2]{haskell}
safeDiv : Int -> Int -> Int!{Exc}
safeDiv a b =
	if b == 0 then
		throw "division by zero!"
	else
		return a / b
\end{minted}

We can call this function like any other function, but no computation will actually be performed.
The effect will remain abstract, we still need to give them a semantics.

\begin{minted}[tabsize=2]{haskell}
result : Int!{Exc}
result = safeDiv 10 2
\end{minted}

In order to actually ``run'' the effect we will need to handle the operations of that effect.
For example, for \ident{Exc} we can write a handler that returns $0$ as a default value if an exception is thrown.

\begin{minted}[tabsize=2]{haskell}
result : Int
result = handle (safeDiv 10 0) {
	throw err k -> return 0
	return v -> return v
} -- results in 0
\end{minted}

For each operation we write a corresponding case in the handler, where we have access to the argument given at operation call and a continuation, which expects a value of the return type of the operation.
There is also a case for values \ident{return}, which gets as an argument the final value of a computation and has the opportunity to modify this value or to do some final computation.
In this case we simply ignore the continuation and exit the computation early with a $0$, we also return any values without modification.
\\\\
We can give multiple ways of handling the same effect.
For example we can also handle the \ident{Exc} effect by capturing the failure or success in a sum type \ident{Either}.

\begin{minted}[tabsize=2]{haskell}
data Either a b = Left a | Right b

result : Either String Int
result = handle (safeDiv 10 0) {
	throw err k -> return (Left err)
	return v -> return (Right v)
} -- results in (Left "division by zero!")
\end{minted}

Here we return early with \ident{Left err} if an error is thrown, otherwise we wrap the resulting value using the \ident{Right} constructor.
\\\\
Another effect we might be interested in is non-determinism.
To model this we define the \ident{Flip} effect interface which has a single operation \ident{flip}, which returns a boolean when called with the unit 
value.

\begin{minted}[tabsize=2]{haskell}
effect Flip {
	flip : () -> Bool
}
\end{minted}

Using the \ident{flip} operation and if-expression we can write non-deterministic computations that can be seen as computation trees where \ident{flip} branches the tree off into two subtrees.
The following program \ident{choose123} non-deterministically returns either a $1$, $2$ or $3$.

\begin{minted}[tabsize=2]{haskell}
choose123 : Bool!{Flip}
choose123 =
	b1 <- flip ();
	if b1 then
		return 1
	else
		b2 <- flip ();
		if b2 then
			return 2
		else
			return 3
\end{minted}

Here the syntax \ident{(x <- c1; c2)} sequences the computations \ident{c1} and \ident{c2} by first performing \ident{c1} and then performing \ident{c2}, where the return value of \ident{c1} can accessed in \ident{x}.
\\\\
Again \ident{choose123} does not actually perform any computation when called, because we have yet to give it a semantics.
We could always return \ident{True} when a \ident{flip} operation is called, in the case of \ident{choose123} this will result in the first branch being picked returning $1$ as the answer.

\begin{minted}[tabsize=2]{haskell}
result : Int
result = handle (choose123) {
	flip () k -> k True
	return v -> return v
} -- returns 1
\end{minted}

Another handler could try all branches returning the greatest integer of all possibilities.

\begin{minted}[tabsize=2]{haskell}
maxresult : Int
maxresult = handle (choose123) {
	flip () k ->
		vtrue <- k True;
		vfalse <- k False;
		return (max vtrue vfalse)
	return v -> return v
} -- returns 3
\end{minted}

Here we first call the continuation \ident{k} with \ident{True} and then with \ident{False}.
The we return the maximum between those results.
\\\\
We could even collect the values from all branches by returning a list.

\begin{minted}[tabsize=2]{haskell}
allvalues : List Int
allvalues = handle (choose123) {
	flip () k ->
		vtrue <- k True;
		vfalse <- k False;
		return vtrue ++ vfalse
	return v -> return [v]
} -- returns [1, 2, 3]
\end{minted}

Again we call the continuation \ident{k} twice, but we append the two results instead.
For the \ident{return} base case we simply wrap the value in a singleton list.
\\\\
Algebraic effects have the nice property that they combine easily.
For example by combining the \ident{Exc} and \ident{Flip} we can implement backtracking, where we choose the first non-failing branch from a computation. For example we can write a function which returns all even sums of the numbers $1$ to $3$ by reusing \ident{choose123}.

\begin{minted}[tabsize=2]{haskell}
evensums123 : Int!{Flip, Exc}
evensums123 =
	n1 <- choose123;
	n2 <- choose123;
	sum <- return (n1 + n2);
	if sum % 2 == 0 then
		return sum
	else
		throw "not even!"
\end{minted}

We implement backtracking in \ident{backtrack} by handling both the \ident{flip} and \ident{throw} operations. For \ident{flip} and the \ident{return} case we do the same as in \ident{allvalues}, calling the continuation \ident{k} with both \ident{True} and \ident{False} and appending the results together. For \ident{throw} we ignore the error message and continuation and exit early with the empty list, this means that branches that results in a failure will not actually return any values.

\begin{minted}[tabsize=2]{haskell}
backtrack : List Int
backtrack () = handle (handle (evensums123) {
	flip () k ->
		vtrue <- k True;
		vfalse <- k False;
		return vtrue ++ vfalse
	return v -> return [v]
}) {
	throw msg k -> return []
	return v -> return v
} -- returns [2, 4, 4, 6]
\end{minted}

We can also handle the effects independently of each other. For example we could implement a partial version of \ident{backtrack} that only handles the \ident{Flip} effect. Any operation that is not in the handler is just passed through.

\begin{minted}[tabsize=2]{haskell}
partlybacktrack : (List Int)!{Exc}
partlybacktrack = handle (evensums123) {
	flip () k ->
		vtrue <- k True;
		vfalse <- k False;
		return vtrue ++ vfalse
	return v -> return [v]
}
\end{minted}

Now we can factor out the \ident{throw} handler into its own function.

\begin{minted}[tabsize=2]{haskell}
fullbacktrack : List Int
fullbacktrack = handle (partlybacktrack) {
	throw msg k -> return []
	return v -> return v
} -- returns [2, 4, 4, 6]
\end{minted}

Algebraic effects always commute, meaning the effects can be handled in any order.
In the backtracking example the order of the handlers does not actually matter, but in general different orders could have different results.
\\\\
Lastly we introduce the \ident{State} effect, which allows us to implement local mutable state.
We restrict ourselves to a state that consists of a single integer value, but in a language with parametric polymorphism a more general state effect could be written.

\begin{minted}[tabsize=2]{haskell}
effect State {
	get : () -> Int
	put : Int -> ()
}
\end{minted}

Our state effect has two operations, \ident{get} and \ident{put}.
The \ident{get} operation allows us to retrieve a value from the state and with the \ident{put} operation we can change the value in the state.
\\\\
We can now implement the familiar ``post increment'' operation as seen in the C programming language.
This function retrieves the current value of the state, increments it by $1$ and returns the previously retrieved value.

\begin{minted}[tabsize=2]{haskell}
postInc : Int!{State}
postInc =
	x <- get ();
	put (x + 1);
	return x
\end{minted}

To implement the semantics of the \ident{State} effect we use parameter-passing similar to how the State monad is implemented in Haskell. We will abstract the implementation of the state handler in a function \ident{runState}.

\begin{minted}[tabsize=2]{haskell}
runState : Int!{State} -> (Int -> (Int, Int))
runState comp = handle (comp) {
	get () k -> return (\s -> (f <- k s; return f s))
	put v k -> return (\s -> (f <- k (); return f v))
	return v -> return (\s -> return (s, v))
}
\end{minted}

\ident{runState} takes a computation that returns an integer and may use the \ident{State} effect, and returns a function that takes the initial value of the state and returns a tuple of the final state and the return value of the computation.
Let us take a look at the \ident{return} case first, here we return a function that takes a state value and returns a tuple of this state and the return value.
For the \ident{get} case we return a function that takes a state value and runs the continuation \ident{k} with this value, giving access to the state at the point where the \ident{get} operation was called. From this continuation we get back another function, which we call with the current state, continuing the computation without changing the state.
The \ident{put} case is similar to the \ident{get} but we call the continuation with the unit value and we continue the computation by calling \ident{f} with the value giving with the \ident{put} operation call.
\\\\
Using state now is as simple as calling \ident{runState}.

\begin{minted}[tabsize=2]{haskell}
stateResult : (Int, Int)
stateResult =
	f <- runState postInc; -- returns a function taking the initial state
	f 42 -- post-increments 42 returning (43, 42)
\end{minted}

Using the state effect we can implement imperative algorithms such as summing a range of numbers.
We first implement a recursive function \ident{sumRangeRec} which uses \ident{State} to keep a running sum.
After we define \ident{sumRange} which calls \ident{sumRangeRec} and runs the \ident{State} effect with $0$ as the initial value.

\begin{minted}[tabsize=2]{haskell}
sumRangeRec : Int -> Int -> Int!{State}
sumRangeRec a b =
	if a > b then
		(_, result) <- get ();
		return result
	else
		x <- get ();
		put (x + a);
		sumRangeRec (a + 1) b
		
sumRange : Int -> Int -> Int
sumRange a b =
	f <- runState (sumRangeRec a b);
	f 0 -- initial sum value is 0
\end{minted}

\section{Static instances}
\label{sec:background-staticinst}

Static instances extend algebraic effects by allowing multiple instances of the same effect to co-exist.
These instances be handled independently of each other.
Operations in such a system are always called on a specific instance and handlers also have to note instance they are handling.
We will write operation calls as \allowhash{\ident{inst#op(v)}} where \ident{inst} is the instance.
Handlers are modified to take an instance parameter as follows \allowhash{\ident{handle#inst(comp) { ... }}}.
\\\\
As an example let us take another look at the \ident{safeDiv} function.

\begin{minted}[tabsize=2]{haskell}
safeDiv : Int -> Int -> Int!{Exc}
safeDiv a b =
	if b == 0 then
		throw "division by zero!"
	else
		return a / b
\end{minted}

We can rewrite this to use static instances by declaring an instance of \ident{Exc} called \ident{divByZero} and calling the \ident{throw} operation on this instance.
Note that in the we now state the instance used instead of the effect, since multiple instances of the same effect could be used and we would like to know which instances exactly.

\begin{minted}[tabsize=2]{haskell}
instance Exc divByZero

safeDiv : Int -> Int -> Int!{divByZero}
safeDiv a b =
	if b == 0 then
		divByZero#throw "division by zero!"
	else
		return a / b
\end{minted}

Imagine we wanted to also throw an exception in the case that the divisor was negative.
Using instances we can easily declare another \ident{Exc} instance, let us call it \ident{negativeDivisor}, and use it in our function.
We also have to modify the type to mention the use of \ident{negativeDivisor}.

\begin{minted}[tabsize=2]{haskell}
instance Exc divByZero
instance Exc negativeDivisor

safeDivPositive : Int -> Int -> Int!{divByZero, negativeDivisor}
safeDivPositive a b =
	if b == 0 then
		divByZero#throw "division by zero!"
	else if b < 0 then
		negativeDivisor#throw "negative divisor!"
	else
		return a / b
\end{minted}

We can now see from the type what kind of exceptions are used in the function.
We can also handle the exceptions independently.
For example we could handle \ident{divByZero} by defaulting to \ident{0}, but leave \ident{negativeDivisor} unhandled.

\begin{minted}[tabsize=2]{haskell}
defaultTo0 : Int!{divByZero, negativeDivisor} -> Int!{negativeDivisor}
defaultTo0 c =
	handle#divByZero (c) {
		throw msg -> return 0
		return v -> return v
	}
\end{minted}

\section{Dynamic instances}
\label{sec:background-dynamicinst}

Having to predeclare every instance we are going to use is very inconvenient, especially when we have effects such as reference cells or communication channels. The global namespace would be littered with all references and channels the program would ever use. Furthermore we do not always know how many references we need. Take for example a function which creates a list of reference cells giving a length $l$. We do not know statically what the length of the list will be and so we do not know ahead how many instances we have to declare.
Furthermore because all the instances would be predeclared some information about the implementation of a function would be leaked to the global namespace. This means it is impossible to fully encapsulate the use of an effect when using static instances.
\\\\
Dynamic instances improve on static instances by allowing instances to be created dynamically.
Instances become first-class values, they can be assigned to variables and passed to functions just like any other value.
We use \ident{new E} to create a new instance of the \ident{E} effect.
The actual implementation of the function can stay exactly the same, as can the handler \ident{defaultTo0}.
We can translate the previous example to use dynamic instances by defining the \ident{divByZero} and \ident{negativeDivisor} as top-level variables and assigning newly created instances to them. We omit type annotation, since there does not exist any type system that can type all usages of dynamic instances.

\begin{minted}[tabsize=2]{haskell}
divByZero = new Exc
negativeDivisor = new Exc

safeDivPositive a b =
	if b == 0 then
		divByZero#throw "division by zero!"
	else if b < 0 then
		negativeDivisor#throw "negative divisor!"
	else
		return a / b
		
defaultTo0 c =
	handle#divByZero (c) {
		throw msg -> return 0
		return v -> return v
	}
\end{minted}

Using locally created instances we can emulate variables as they appear in imperative languages more easily.
We can implement the factorial function in an imperative style using a locally created \ident{State} instance.
The \ident{factorial} function computes the factorial of the paramter \ident{n} by creating a new \ident{State} instance named \ident{ref} and calling the helper function \ident{factorialLoop} with \ident{ref} and \ident{n}.
The base case of \ident{factorialLoop} retrieves the current value from \ident{ref} and returns it.
In the recursive case of \ident{factorialLoop} the value in \ident{ref} is modified by multiplying it by \ident{n} and then we continue by recursing with \ident{n - 1}.
The call to \ident{factorialLoop} in \ident{factorial} is wrapped in the \ident{State} handler explained earlier, chosing \ident{1} as the initial value of \ident{ref}.
\ident{factorial} thus computes the factorial of a number by using a locally created instance, but the use of this instance or the \ident{State} effect in general never escapes the function, it is completely encapsulated.

\pagebreak
\begin{minted}[tabsize=2]{haskell}
factorialLoop ref n =
	if n == 0 then
		ref#get ()
	else
		x <- ref#get();
		ref#put (x * n);
		factorialLoop ref (n - 1) 

factorial n =
	ref <- new State;
	statefn <- handle#ref (factorialLoop ref n) {
		get () k -> return (\s -> (f <- k s; return f s))
		put v k -> return (\s -> (f <- k (); return f v))
		return v -> return (\s -> return v)
	};
	statefn 1 -- use 1 as the initial value of ref
\end{minted}

Next we will implement references more generally similar to the ones available in Standard ML~\autocite{standardml}, in our case specialized to \ident{Int}.
In the previous example we see a pattern of creating a \ident{State} instance and then calling some function with it wrapped with a handler.
This is the pattern we want to use when implementing references.
To implement this pattern more generally this we first introduce a new effect named \ident{Heap}.
\ident{Heap} has one operation called \ident{ref} which takes an initial value \ident{Int} and returns a \ident{State} instance.
\ident{Heap} can be seen as a collection of references.
We then define a handler \ident{runRefs} which takes a \ident{Heap} instance and a computation, and creates \ident{State} instances for every use of \ident{ref}. After we call the continuation with the newly created instance and wrap this call in the usual \ident{State} handler, giving the argument of \ident{ref} as the initial value.

\begin{minted}[tabsize=2]{haskell}
effect Heap {
	ref : Int -> Inst State
}

runRefs inst c =
	handle#inst (c) {
		ref v k ->
			r <- new State;
			statefn <- handle#r (k r) {
				get () k -> return (\s -> (f <- k s; return f s))
				put v k -> return (\s -> (f <- k (); return f v))
				return v -> return (\s -> return v)
			};
			statefn v 
		return v -> return v
	}
\end{minted}

By calling \ident{runRefs} at the top-level we will have the same semantics for references as Standard ML.
In the following example we create two references and swap their values using a \ident{swap} function.
First \ident{main} creates a new \ident{Heap} instance \ident{heap} and then calls \ident{runRefs} with this instance.
The computation given to \ident{runRefs} is the function \ident{program} called with \ident{heap}.

\pagebreak
\begin{minted}[tabsize=2]{haskell}
swap r1 r2 =
	x <- r1#get ();
	y <- r2#get ();
	r1#put(y);
	r2#put(x)
	
program heap =
	r1 <- heap#ref 1;
	r2 <- heap#ref 2;
	swap r1 r2;
	x <- r1#get ();
	y <- r2#get ();
	return (x, y)
	
main =
	heap <- new Heap;
	runRefs heap (program heap) -- returns (2, 1)
\end{minted}

In the Haskell programming language the ST monad~\autocite{runst} can be used to implement algorithms that internally use mutable state.
The type system, using the \ident{runST} function, will make sure that the mutable state does not leak outside of the function.
For example the following function \ident{fibST} implements the Fibonacci function in constant space by creating two mutable references.

\begin{minted}[tabsize=2]{haskell}
fibST :: Integer -> Integer
fibST n = 
    if n < 2 then
      n
    else runST $ do
        x <- newSTRef 0
        y <- newSTRef 1
        fibST' n x y

    where fibST' 0 x _ = readSTRef x
          fibST' n x y = do
              x' <- readSTRef x
              y' <- readSTRef y
              writeSTRef x y'
              writeSTRef y $! x' + y'
              fibST' (n - 1) x y
\end{minted}

Using dynamic instances we can implement the same algorithm, named \ident{fib} below.
Our \ident{fib} takes a parameter \ident{n} and returns the \ident{n}th Fibonacci number.
First we check if \ident{n} is smaller than 2, in which case we can return \ident{n} as the result, since $n$th Fibonacci number is $n$, if $n < 2$.
Else we create a new \ident{Heap} instance named \ident{heap} and use the \ident{runRefs} function defined earlier to run a computation on this heap.
We create two \ident{State} instances on \ident{heap}, \ident{x} and \ident{y} initialized with \ident{0} and \ident{1} respectively and call the auxillary function \ident{fibRec} with \ident{n} and the two instances \ident{x} and \ident{y}.
\ident{fibRec} implements the actual algorithm.
It works by (recursively) looping on \ident{n}, subtracting by \ident{1} each recursive call.
\ident{x} and \ident{y} store the current and next Fibonacci respectively and each loop they are moved one Fibonacci number to the right.
When \ident{n} is \ident{0} we know \ident{x} contains the \ident{n}th (for the initial value of \ident{n}) Fibonacci number and we can just get the current value from \ident{x} and return it.
Even though this algorithm uses the \ident{Heap} and \ident{State} effects, their uses are completely encapsulated by the \ident{fib} function.
The \ident{fib} function does not leak the fact that it's using those effects to implement the algorithm.

\begin{minted}[tabsize=2]{haskell}
fib n =
  if n < 2 then
    n
  else
    heap <- new Heap;
    runRefs heap (
      x <- heap#ref 0;
      y <- heap#ref 1;
      fibRec n x y
    )

fibRec n x y =
  if n == 0 then
    x#get ()
  else
    x' <- x#get ();
    y' <- y#get ();
    x#put(y');
    y#put(x' + y');
    fibRec (n - 1) x y
\end{minted}

Dynamic instances have one big problem though: they are too dynamic. Similar to how in general it is undecidable to know whether a reference has escaped its scope, it is also not possible to know whether an instance has a handler associated with it. This makes it hard to think of a type system for dynamic instances which ensures that there are no unhandled operations. Earlier versions of the Eff programming language~\autocite{eff1} had dynamic instances but its type system underapproximated the uses of dynamic instances which meant you could still get a runtime error if any operation calls were left unhandled.
