{
% types
\newcommand\ty[0]{\tau}
\newcommand\tunit[0]{()}
\newcommand\tarr[2]{#1 \rightarrow #2}
\newcommand\thandler[2]{#1 \Rightarrow #2}
\newcommand\tforall[3]{\forall(#1:#2) . #3}

% computation type
\newcommand\cty[0]{\underline{\ty}}
\newcommand\aty[2]{#1 \; ! \; #2}
\newcommand\texists[3]{\exists(#1:#2) . #3}
\newcommand\texistss[2]{\exists \overrightarrow{#1} . #2}

% values
\newcommand\val[0]{\nu}
\newcommand\vunit[0]{()}
\newcommand\vinst[1]{\mathsf{inst}(#1)}
\newcommand\vabst[3]{\Lambda(#1:#2) . #3}
\newcommand\vabs[2]{\lambda #1 . #2}
\newcommand\vappt[2]{#1 \; [ #2 ]}
\newcommand\vhandler[1]{\textit{handler} \; \{#1\}}
\newcommand\vhandleri[2]{\textit{handler} ( #1 ) \; \{#2\}}
\newcommand\vhandlerc[0]{\vhandler{
	\textit{return} \; x \rightarrow \comp,
	\op_1(x ; k) \rightarrow \comp,
	...,
	\op_n(x ; k) \rightarrow \comp
}}
\newcommand\vhandlerci[1]{\vhandleri{#1}{
	\textit{return} \; x \rightarrow \comp,
	\op_1(x ; k) \rightarrow \comp,
	...,
	\op_n(x ; k) \rightarrow \comp
}}

% computations
\newcommand\comp[0]{c}
\newcommand\creturn[1]{\textit{return} \; #1}
\newcommand\capp[2]{#1 \; #2}
\newcommand\cdo[3]{#1 \leftarrow #2 ; #3}
\newcommand\cop[4]{#1(#2 ; #3 . #4)}
\newcommand\copi[5]{#1 \# #2(#3 ; #4 . #5)}
\newcommand\chandle[2]{\textit{with} \; #1 \; \textit{handle} \; #2}
\newcommand\cnew[1]{\textit{new} \; #1}
\newcommand\cunpack[4]{(#1, #2) \leftarrow #3 ; #4}

In this chapter we will show the basics of algebraic effects and handlers. We will start with the simply-typed lambda calculus and add algebraic effects and instances to it. We end with dynamic instances and show why a type system for them is difficult to implement.

\section{Fine-grained call-by-value simply-typed lambda calculus}

\subsection{Intro}


\subsection{Syntax}
\begin{align*}
	\ty \Coloneqq 	& 												\tag{value types} \\
									& \tunit											\tag{unit type} \\
									& \tarr{\ty}{\cty}				\tag{type of functions} \\
	\cty \Coloneqq 	& 											\tag{computation types} \\
									& \ty										\tag{value type} \\
	\val \Coloneqq			&											\tag{values} \\
									& x, y, z, k								\tag{variables} \\
									& \vunit									\tag{unit value} \\
									& \vabs{x}{\comp}					\tag{abstraction} \\
	\comp \Coloneqq		&											\tag{computations} \\
									& \creturn{\val}						\tag{return value as computation} \\
									& \capp{\val}{\val}					\tag{application} \\
									& \cdo{x}{\comp}{\comp}			\tag{sequencing} \\
\end{align*}

\subsection{Semantics}

\subsection{Type system}

\section{Algebraic effects}

\subsection{Intro}

\subsection{Syntax}

\subsection{Semantics}

\subsection{Type system}

\section{Static instances}

\subsection{Intro}

\subsection{Syntax}

\subsection{Semantics}

\subsection{Type system}

\section{Dynamic instances}

\subsection{Intro}

\subsection{Syntax}

\subsection{Semantics}

\subsection{Type system (discussion, problems)}
}