{
% types
\newcommand\ty[0]{\tau}
\newcommand\tunit[0]{()}
\newcommand\tarr[2]{#1 \rightarrow #2}
\newcommand\thandler[2]{#1 \Rightarrow #2}
\newcommand\tforall[3]{\forall(#1:#2) . #3}

% computation type
\newcommand\cty[0]{\underline{\ty}}
\newcommand\aty[2]{#1 \; ! \; #2}
\newcommand\texists[3]{\exists(#1:#2) . #3}
\newcommand\texistss[2]{\exists \overrightarrow{#1} . #2}

% values
\newcommand\val[0]{\nu}
\newcommand\vunit[0]{()}
\newcommand\vinst[1]{\mathsf{inst}(#1)}
\newcommand\vabst[3]{\Lambda(#1:#2) . #3}
\newcommand\vabs[2]{\lambda #1 . #2}
\newcommand\vappt[2]{#1 \; [ #2 ]}
\newcommand\vhandler[1]{\textit{handler} \; \{#1\}}
\newcommand\vhandleri[2]{\textit{handler} ( #1 ) \; \{#2\}}
\newcommand\vhandlerc[0]{\vhandler{
	\textit{return} \; x \rightarrow \comp,
	\op_1(x ; k) \rightarrow \comp,
	...,
	\op_n(x ; k) \rightarrow \comp
}}
\newcommand\vhandlerci[1]{\vhandleri{#1}{
	\textit{return} \; x \rightarrow \comp,
	\op_1(x ; k) \rightarrow \comp,
	...,
	\op_n(x ; k) \rightarrow \comp
}}

% computations
\newcommand\comp[0]{c}
\newcommand\creturn[1]{\mathsf{return} \; #1}
\newcommand\capp[2]{#1 \; #2}
\newcommand\cdo[3]{#1 \leftarrow #2 ; #3}
\newcommand\cop[4]{#1(#2 ; #3 . #4)}
\newcommand\copi[5]{#1 \# #2(#3 ; #4 . #5)}
\newcommand\chandle[2]{\textit{with} \; #1 \; \textit{handle} \; #2}
\newcommand\cnew[1]{\textit{new} \; #1}
\newcommand\cunpack[4]{(#1, #2) \leftarrow #3 ; #4}

In this chapter we will show the basics of algebraic effects and handlers. We will start with the simply-typed lambda calculus (\cref{section:stlc}) and add algebraic effects and instances to it. We end with dynamic instances and show why a type system for them is difficult to implement.

\section{Simply-typed lambda calculus} \label{section:stlc}

As our base language we will take the fine-grained call-by-value simply-typed lambda calculus (FG-STLC) \cite{fg-stlc}.
This system is a version of the simply-typed lambda calculus with a syntactic distinction between values and computations.
Because of this distinction there is exactly one evaluation order.
In a system with side effects the evaluation order is very important since a different order could have a different result.
Having the evaluation order be apparent from the syntax is thus a good choice for a system with algebraic effects.

%\subsection{Syntax}

\begin{figure}
\caption{Syntax of the fine-grained lambda calculus}
\centering
\fbox{
\begin{minipage}{10 cm}
\begin{align*}
	\val \Coloneqq	&													\tag{values} \\
									& x, y, z, k							\tag{variables} \\
									& \vabs{x}{\comp}					\tag{abstraction} \\
									& \vunit									\tag{unit value} \\
	\comp \Coloneqq&													\tag{computations} \\
									& \creturn{\val}					\tag{return value as computation} \\
									& \capp{\val}{\val}				\tag{application} \\
									& \cdo{x}{\comp}{\comp}		\tag{sequencing} \\
\end{align*}
\label{fig:syntax-stlc}
\end{minipage}
}
\end{figure}

The terms are shown in Figure \ref{fig:syntax-stlc}.
The terms are split in to values and computations.
Values are pieces of data that have no effects, while computations are terms that may have effects.

\textbf{Values} We have $x$, $y$, $z$, $k$ ranging over variables, where we will use $k$ for variables that denote continuations later on.
We also have the usual abstractions though note that the body is a computation.
To keep things simple we take unit as our only base value, this because adding more base values will not complicate the theory.
Using the unit value we can also delay computations by wrapping them in an abstraction that takes a unit value.
\\\\
\textbf{Computations} For any value $\val$ we have $\creturn{\val}$ for the computation that simply returns a value without performing any effects. We have function application $(\capp{\val}{\val})$, where both the function and argument have to be values. Sequencing computations is done with $(\cdo{x}{\comp}{\comp})$. Normally in the lambda calculus the function and the argument in an application could be any term and so a choice would have to be made in what order these have to be evaluated or whether to evaluate the argument at all before substitution. In the fine-grained calculus both the function and argument in $(\capp{\val}{\val})$ are values so there's no choice of evaluation order. The order is made explicit by the sequencing syntax $(\cdo{x}{\comp}{\comp})$.

%\subsection{Semantics}

\begin{figure}
\caption{Semantics of the fine-grained lambda calculus}
\centering
\fbox{
\begin{minipage}{11 cm}
\[\inferrule{
}{
	\capp{(\vabs{x}{\comp})}{\val} \rightsquigarrow \comp[x := \val]
}\quad(\footnotesize\textsc{STLC-S-App})\]

\[\inferrule{
}{
	(\cdo{x}{\creturn{\val}}{\comp}) \rightsquigarrow \comp[x := \val]
}\quad(\footnotesize\textsc{STLC-S-SeqReturn})\]

\[\inferrule{
	\comp_1 \rightsquigarrow \comp_1'
}{
	(\cdo{x}{\comp_1}{\comp_2}) \rightsquigarrow (\cdo{x}{\comp_1'}{\comp_2})
}\quad(\footnotesize\textsc{STLC-S-Seq})\]
\label{fig:semantics-stlc}
\end{minipage}
}
\end{figure}

\textbf{Semantics}
The small-step operational semantics is shown in Figure \ref{fig:semantics-stlc}.
The relation $\rightsquigarrow$ is defined on computations, where the $\comp \rightsquigarrow \comp'$ means $\comp$ reduces to $\comp'$ in one step.
These rules are a fine-grained approach to the standard reduction rules of the simply-typed lambda calculus.
In {\footnotesize\textsc{STLC-S-App}} we apply a lambda abstraction to a value argument, by substituting the value for the variable $x$ in the body of the abstraction.
In {\footnotesize\textsc{STLC-S-SeqReturn}} we sequence a computation that just returns a value in another computation by substituting the value for the variable $x$ in the computation.
Lastly, in {\footnotesize\textsc{STLC-S-Seq}} we can reduce a sequence of two computations, $\comp_1$ and $\comp_2$ by reducing the first, $\comp_1$.

% say something about transitive reflexive closure

%\subsection{Type system}

\begin{figure}
\caption{Types of the fine-grained simply-typed lambda calculus}
\centering
\fbox{
\begin{minipage}{8 cm}
\begin{align*}
	\ty \Coloneqq 	& 												\tag{value types} \\
									& \tunit									\tag{unit type} \\
									& \tarr{\ty}{\cty}				\tag{type of functions} \\
	\cty \Coloneqq 	& 												\tag{computation types} \\
									& \ty											\tag{value type} \\
\end{align*}
\label{fig:types-stlc}
\end{minipage}
}
\end{figure}

\textbf{Types}
Next we give the \emph{types} in Figure \ref{fig:types-stlc}.
Similar to the terms we split the syntax into value and computation types.
Values are typed by value types and computations are typed by computation types.
A value type is either the unit type $\tunit$ or a function type with a value type $\ty$ as argument type and a computation type  $\cty$ as return type.\\
For the simply-typed lambda calculus a computation type is simply a value type, but when we add algebraic effects computation types will become more meaningful by recording the effects a computation may use.

\begin{figure}
\caption{Typing rules of the fine-grained simply-typed lambda calculus}
\centering
\fbox{
\begin{minipage}{14 cm}
\[\inferrule{
	\Gamma[x] = \ty
}{
	\Gamma \vdash x : \ty
}\quad(\footnotesize\textsc{STLC-T-Var})\]

\[\inferrule{
}{
	\Gamma \vdash \vunit : \tunit
}\quad(\footnotesize\textsc{STLC-T-Unit})\]

\[\inferrule{
	\Gamma, x : \ty_1 \vdash \comp : \cty_2
}{
	\Gamma \vdash \vabs{x}{\comp} : \tarr{\ty_1}{\cty_2}
}\quad(\footnotesize\textsc{STLC-T-Abs})\]

\[\inferrule{
	\Gamma \vdash \val : \ty
}{
	\Gamma \vdash \creturn{\val} : \cty
}\quad(\footnotesize\textsc{STLC-T-Return})\]

\[\inferrule{
	\Gamma \vdash \val_1 : \tarr{\ty_1}{\cty_2}\\
	\Gamma \vdash \val_2 : \ty_1
}{
	\Gamma \vdash \capp{\val_1}{\val_2} : \cty_2
}\quad(\footnotesize\textsc{STLC-T-App})\]

\[\inferrule{
	\Gamma \vdash \comp_1 : \cty_1\\
	\Gamma, x : \ty_1 \vdash \comp_2 : \cty_2
}{
	\Gamma \vdash (\cdo{x}{\comp_1}{\comp_2}) : \cty_2
}\quad(\footnotesize\textsc{STLC-T-Seq})\]
\label{fig:typing-stlc}
\end{minipage}
}
\end{figure}

\textbf{Typing rules}
Finally we give the typing rules in Figure \ref{fig:typing-stlc}.
We have a typing judgment for values $\Gamma \vdash \val : \ty$ and a typing judgment for computations $\Gamma \vdash \comp : \cty$.
In both these judgments the context $\Gamma$ assigns value types to variables.\\
The rules for variables ({\footnotesize\textsc{STLC-T-Var}}), unit ({\footnotesize\textsc{STLC-T-Unit}}), abstractions ({\footnotesize\textsc{STLC-T-Abs}}) and applications ({\footnotesize\textsc{STLC-T-App}}) are the standard typing rules of the simply-typed lambda calculus.
For $\creturn{\val}$ ({\footnotesize\textsc{STLC-T-Return}}) we simply check the type of $\val$.
For the sequencing of two computations $(\cdo{x}{\comp_1}{\comp_2})$ ({\footnotesize\textsc{STLC-T-Seq}}) we first check the type of $\comp_1$ and then check $\comp_2$ with the type of $\comp_1$ added to the context for $x$.

% \subsection{Examples}

To show the explicit order of evaluation we will translate the following program from the simply-typed lambda calculus into its fine-grained version:\\
\[\capp{\capp{f}{x}}{y}\]\\
Here we have a choice of whether to first evaluate $x$ or $y$ and whether to evaluate $(\capp{f}{x})$ before evaluating $y$.
In the fine-grained system this choice is made explicit by the sequencing of computations.
As such we can write down three variants for the above program, each having a different evaluation order.
In the presence of effects all three may have different results.

$x$ before $y$, $y$ before $(\capp{f}{x})$: \\\indent
$\cdo{x'}{x}{\cdo{y'}{y}{\cdo{g}{(\capp{f}{x'})}{(\capp{g}{y'})}}} $\\
$y$ before $x$, $y$ before $(\capp{f}{x})$:\\\indent
$\cdo{y'}{y}{\cdo{x'}{x}{\cdo{g}{(\capp{f}{x'})}{(\capp{g}{y'})}}} $\\
$x$ before $y$, $y$ before $(\capp{f}{x})$:\\\indent
$\cdo{x'}{x}{\cdo{g}{(\capp{f}{x'})}{\cdo{y'}{y}{(\capp{g}{y'})}}} $\\

% \subsection{Theorems}
\textbf{Type safety}
\[
	\mathsf{if}\;
	\cdot \vdash \comp : \cty \;\mathsf{and}\;
	\comp \rightsquigarrow_* \comp' \;\mathsf{then}\;
	\comp' = \creturn{\val} \;\mathsf{and}\;
	\cdot \vdash \val : \ty
\]

\newpage
\section{Algebraic effects}

\subsection{Intro}
Explain:
\begin{itemize}
	\item What are algebraic effects and handlers
	\item Why algebraic effects
	\begin{itemize}
		\item easy to use
		\item can express often used monads
		\item composable
		\item always commuting
		\item modular (split between computations and handlers)
	\end{itemize}
\end{itemize}

\subsection{Syntax}

\subsection{Semantics}

\subsection{Type system}

\subsection{Examples}
Show flip (non-determinism) and state examples.

\newpage
\section{Static instances}

\textit{Should I even mention static instances?}

\subsection{Intro}
Explain:
\begin{itemize}
	\item Show problems with wanting to use multiple state instances
	\item What are static instances
	\item Show that static instances partially solve the problem
\end{itemize}

\subsection{Syntax}

\subsection{Semantics}

\subsection{Type system}

\subsection{Examples}
Show state with multiple static instances (references).

\newpage
\section{Dynamic instances (untyped)}

\subsection{Intro}
Explain:
\begin{itemize}
	\item Show that static instances require pre-defining all instances on the top-level.
	\item Static instances not sufficient to implement references.
	\item Show that dynamic instances are required to truly implement references.
	\item Show more uses of dynamic instances (file system stuff, local exceptions)
	\item No type system yet.
\end{itemize}

\subsection{Syntax}

\subsection{Semantics}

\subsection{Examples}
Show untyped examples.
\begin{itemize}
	\item Local exceptions
	\item ML-style references
\end{itemize}

\subsection{Type system (discussion, problems)}
Show difficulty of implementing a type system for this.
}