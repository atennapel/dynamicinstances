\chapter{\label{chap:conclusion}Conclusion and future work}

We conclude with a brief discussion of what we presented in this thesis and discuss possible future work.

\section{Conclusion}
In \cref{chap:algintro} we have seen that algebraic effects and handlers are a composable approach to programming with side-effects.
Using algebraic effects effects we can keep functions pure until we handle the effects within them.
We have also seen that we are unable to express mutable references within algebraic effects and handlers.
Dynamic instances, as introduced by Bauer and Pretnar in Eff~\autocite{eff1}, allow the programmer to dynamically generate instances of effects.
Using dynamic instances one can implement dynamic effects such as mutable references and the dynamic opening of channels.
Unfortunately the type system of Eff under-approximates the uses of effects, which can lead to runtime errors.
\\\\
In \cref{chap:langintro} we presented a new language \lang{} with algebraic effects and handlers, and dynamic instances.
Using a notion of effect scopes we are able to safely use dynamically created instances, ensuring that all operation calls are handled.
We have shown how we can implement mutable references and vectors in \lang{}.
\\\\
In \cref{chap:algtheory} we presented formal accounts of algebraic effects and handler, with and without static instances, giving a type system and operational semantics.
We have formalized these systems and proven type safety in Coq.
\\\\
Finally, in \cref{chap:langtheory} we have presented a type system and operational semantics for the core language of \lang{}.
We ended the chapter with a discussion on the difficulties of proving type safety for \lang{} and gave a possible approach that might allow us to make progress on these proofs.
\\\\
In this thesis we have shown that we can safely combine algebraic effects with a restricted form of dynamic instances by giving an explicit scope for the use of an instance.
Using the notion of an effect scope and effect scope variables we can ensure that no operation call will be left unhandled and we can avoid runtime errors.
We have also discussed the difficulties in proving type safety for \lang{}.
By enabling the definition of dynamic effects, such as mutable references, algebraic effects and handlers can be useful in more situations.
 
\section{Future work}

\paragraph{Mechanization}
We currently have formalized the system with static instances of \cref{sec:background-algeff} and \cref{sec:background-staticinst} and have proven type safety in Coq\footnote{https://github.com/atennapel/dynamicinstances}.
This formalization is briefly discussed in \cref{sec:formalization}.
Due to time constraints we were unable to also provide a formalization for \lang{}.
It would be beneficial to also formalize the syntax, semantics and typing rules \lang{} and to prove type safety, in order to gain more certainty the system is safe.
\\\\
We have implemented a prototype of \lang{} in Haskell\footnote{https://github.com/atennapel/dynamicinstances/tree/master/hs}.
We implemented the typing rules in a bidirectional~\autocite{bidirectionaltyping} style.
We also implemented the small-step operation semantics.
Using the prototype we can typecheck and run \lang{} programs and verify our ideas.
\\\\
There are many kinds of features possible which increase the expressiveness and guarantees of \lang{}.
We will now discuss possible extensions to \lang{}.

\paragraph{Parametric polymorphism over any type.}
In order to keep the system simple and to only focus on the novel elements, \lang{} only supports parametric polymorphism over effect scopes.
It would be very useful in practice to allow quantification over any type.
We do not think that adding this will interfere with the other elements of the system.

\paragraph{Polymorphic effects.}
Having added polymorphism over any type it makes sense to also allow for polymorphic effects.
In our examples we have defined a \ident{State} effect with \ident{Int} values.
In order to avoid having to define a seperate effect for each type we would like to keep in our state, we could allow for effects to have type parameters.
For example, we could define a polymorphic \ident{State} effect like:
\begin{minted}[tabsize=2]{haskell}
effect State t {
	get : () -> t
	put : t -> ()
}
\end{minted}

Using this effect we can have fully polymorphic mutable references.
For example, the type of a reference holding an integer value would be \ident{Inst s (State Int)}.
Given that each reference carries the type of the value in the reference, \ident{get} and \ident{put} can be type safe.

\iffalse
\paragraph{Polymorphic operations.}
Polymorphism in operations can also be useful for some effects.
For example an exception effect:
\begin{minted}[tabsize=2]{haskell}
effect Exception {
	throw : forall t. String -> t
}
\end{minted}
Here, because we do not expect to continue a computation after an exception has been throw, we can let the \ident{throw} operation return any type.
This makes it convenient to throw errors, for example in an if-expression where the type of both branches have to match.
Adding polymorphic operations takes some more thought because in the case of operations like \ident{throw} we cannot allow a handler to call the continuation.
\\\\
Another example is if we define an effect that is suppose to wrap the creation of a \ident{State} instance:
\begin{minted}[tabsize=2]{haskell}
effect CreateRef {
	ref : forall s. Int -> Inst s State
}
\end{minted}
Here, we want to call the \ident{ref} operation to create a new instance.

\fi

\paragraph{Improved effect annotations.}
Currently the effect annotation of a computation type is a set of effect scopes.
We could make these annotations more precise by also noting which effects are used on each scope.
For example, from \ident{Int!{s1, s2}} to \ident{Int!{{State, Flip}@s1, {Rng}@s2}}.
We could also allow users to restrict which effects occur on which effect scope in this way, giving more static guarantees.
We do not see any difficulty in extending the annotations in this way.

\paragraph{Combine with regular algebraic effects and handlers.}
In \lang{} handlers are given when creating instances.
These handlers are necessary in order to make sure that every instance has a handler, which completely handles the effects of that specific instance.
In regular algebraic effects (\cref{sec:background-algeff}) operation calls can be called anywhere and can also be handled anywhere higher up.

These are two different ways of using algebraic effects.
We could combine \lang{} with the regular algebraic effects and handlers.
For example:
\begin{minted}[tabsize=2]{haskell}
combination : forall s. (Inst s State) -> ()!{s, State}
combination [s] r =
	x <- #get();
	r#put(x)
\end{minted}

The function \ident{combination} takes as argument a \ident{State} instance on some effect scope \ident{s}.
We then call the \ident{get} operation, but not on an instance and call this value \ident{x}.
We then store \ident{x} in the reference argument \ident{r}, by calling the \allowhash{\ident{r#put}}.
From the effect annotation on the type of \ident{combination} we can see we are both using the effect scope \ident{s} and \ident{State} without a scope.
While \ident{r} already has a handler associated with it, because one has to be given when creating it, the \allowhash{\ident{#get}} operation does not.
We still have to give a handler for \allowhash{\ident{#get}} higher up, like one would do with regular algebraic effects and handlers.
Effect interfaces can be used for both systems as we do not change these from the regular algebraic effects system in \lang{}.
It is not clear how difficult it is to combine these two systems as they can interact, regular operations can be called within an effect scope.

\paragraph{Global scope} \label{sec:globalscopes}
Usually mutable references are globally scoped, meaning they are valid for the entire program.
In \lang{} we have to explicitly scope instances using \ident{runscope}.
In order to fully emulate global mutable references we could add a special scope location \ident{global}.
Instances created on the \ident{global} scope are globally scoped and can be used anywhere.
We can modify the \ident{ref} function from \cref{sec:lang-basics} to create globally scoped mutable references.
\begin{minted}[tabsize=2]{haskell}
globalref : Int -> (Inst global State)!{global}
globalref v =
	new State@global {
		get () k -> \st -> k st st
		put st' k -> \st -> k () st'
		return x -> \st -> return x
		finally f -> f v
	} as x in return x
\end{minted}

Notice that we no longer need \ident{forall s.} in the type.
By combining globally scoped instances and polymorphic effects we would be able to emulate fully polymorphic mutable references, as seen in the ML programming language.
We would need a special $\mathsf{runglobalscope}$ construct in the semantics which always surrounds the entire program and handles any global instances.
Care would have to be taken to ensure global instances cannot escape, we have to still make sure their operations are always handled.
Questions remain on what limitations we need on the global handlers.
Should we be allowed to call other effects in the handlers?
Should we be allowed to invoke the continuation zero or multiple times?
This idea is similar to the concept of resources in the Eff programming language, as discussed in the related work in \cref{sec:eff}.
Handlers on the global scope are similar to default handlers.
